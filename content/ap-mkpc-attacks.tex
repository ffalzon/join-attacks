% \begin{lemma}
% 	\label{lem:dictinfer_MR}
% 	Let \setX{} and \setY{} be two sets of records and let $(\leakX, \leakY) \sample \leak(\setX, \setY)$.
% 	Moreover, let $f_C: \Ispace{} \to \G$ be the hiding function sampled in \LMKPM{} to produce \leakX{} and \leakY.
% 	If $D$ is a dictionary mapping $f_C(\id)$ to \id{} for all $\id \in \Ispace{\setX}$,
% 	then $\dictInfer(D, \leakY)$ (\cref{fig:dict_infer}) produces a maximum reconstruction of \setY{} with respect to \setX{}.
% \end{lemma}
% \begin{proof}
% 	Let $\mathfrak{R}$ be the multiset produced by \dictInfer.
% 	Recall from \cref{sec:notation} that we consider different instantiations 
% 	of the same record in $\mathfrak{R}$ as distinct elements.
% 	Note that for each $y \in \leakY$ exactly one reconstructed record $r$
% 	is added to $\mathfrak{R}$.
% 	Let $d: \mathfrak{R} \to \leakY$ denote the function mapping these reconstructed records 
% 	to the corresponding $y$. Clearly, $d$ is a bijection.

% 	We now define an injective function $\varphi: \mathfrak{R} \to \setY$ and prove that it satisfies the three conditions for Maximum Set Reconstruction (\cref{def:max_recons}).
% 	For any record $r \in \mathfrak{R}$ let $(d_1, \dots, d_k)=d(r) \in \leakY$ and let
% 	$$\varphi(r) := (f_C^{-1}(d_1), \dots, f_C^{-1}(d_k))$$
	
% 	Note that $f_C^{-1}(d_j)$ exists for all $r$ and $j$, since $d(r) \in \leakY$ by definition.
% 	Moreover, since $f_C$ is injective, $f_C^{-1}$ is injective for elements in $f_C(\Ispace{})$.
% 	$\varphi$ is injective since $d$ and $f_C^{-1}$ are injective.
% 	We show the three properties from \cref{def:max_recons} separately:
% 	\begin{enumerate}
% 		\item Let $r\in \mathfrak{R}$, $k := |r|$ and $i \leq |r|$. 
% 		Moreover, let $(d_1, \dots, d_{k'}) = d(r)$ for some $k' \geq k$.
% 		By definition of $d$ and by construction of $r$ (line \ref{lin:substitute_add_id}), 
% 		there exists some $j^*\in[k']$ such that $r[i] = D[d_{j^*}]$.
% 		Suppose for a contradiction that $r[i] \not \in \varphi(r)$, that is,
% 		we have $r[i] \neq f_C^{-1}(d_j)$ for all $j \in [k']$.
% 		But since for all $j \in [k']$ with $d_j \in D$ we have $f_C^{-1}(d_j) = D[d_j]$,
% 		we also have that $r[i] \neq D[d_j]$, i.e., there is no $j$ such that $r[i] = D[d_j]$. 
% 		This is a direct contradiction to $r[i] = D[d_{j^*}]$.
% 		% However, if this holds, then the condition on line \ref{lin:substitute_membership_check} 
% 		% would not have been true for any $j\in[k']$ and thus $r[i]$ would not have been added, 
% 		% thus resulting in a contradiction. 
% 		\item Let $r\in \mathfrak{R}$ and $(d_1, \dots, d_{k'}) = d(r)$ for some $k' \in \N$.
% 		Let $\id \in \varphi(r)$, 
% 		i.e., there is some $i \in [k']$ such that $\id = f_C^{-1}(d_i)$.
% 		If $\id \in \Ispace{\setX}$, then by assumption we have $f_C(\id) \in D$ and $D[f_C(\id)] = \id$.
% 		Since $\id = f_C^{-1}(d_i)$, we have $f_C(\id) = f_C(f_C^{-1}(d_i)) = d_i$ 
% 		and thus $d_i \in D$. Therefore, the condition on line~\ref{lin:substitute_membership_check} is met and
% 		$D[d_i] = D[f_C(\id)] = \id$ is added to $r$.
% 		\item Holds trivially, since we have $\varphi(\mathfrak{R}) = \setY$ by the definition of $d$. 
% 	\end{enumerate}
% The lemma thus follows.
% \end{proof}

% \begin{lemma}
% 	For any target set \setT{} and victim set \setV{}, 
% 	$\toyattack^{\LMKPM(\cdot, \setV)}(\setT)$ from \cref{fig:toy_attack}
% 	produces a maximum reconstruction of \setV{} with respect to \setT{}.
% \end{lemma}
% \begin{proof}
% 	It is easy to see that the record $r$ constructed in lines \ref{lin:baseline_r_start}~-~\ref{lin:baseline_r_end} contains all identifiers from 
% 	$\Ispace{\setT}$ and therefore $\Ispace{\setT} = \Ispace{\setT'}$. 
% 	It therefore suffices to show that $\toyattack$ produces a maximum reconstruction of $\setV$ w.r.t. $\setT'$.
	
% 	Let $f_C$ be the identifier hiding function sampled during the invocation of 
% 	\LMKPM{} in line \ref{lin:baseline_eval},  
% 	let $D$ be the dictionary constructed in lines \ref{lin:baseline_D_start} and \ref{lin:baseline_D_end} 
% 	and let $k = |r|$.
% 	By definition of \LMKPM{}, we have $r_e = (f_C(r[1]), \dots, f_C(r[k]))$
% 	and therefore $D[f_C(r[i])] = D[r_e[i]] = r[i]$ for all $i \in [k]$.
% 	Since $r$ contains all identifiers from $\Ispace{\setT'}$, the claim follows from \cref{lem:dictinfer_MR}.
% \end{proof}