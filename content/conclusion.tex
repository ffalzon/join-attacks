\section{Discussion}

\subsection{Extensions}

\subsection{Mitigations}

% The mitigation of the record enumeration attack is clearly more 
% involved than with the baseline attack,
% since MK-PrivateID heavily relies on the preservation of the identifier order 
% and the deterministic hiding of identifiers
% in order to compute its multi-key matching logic (see \cref{fig:match_logic}).
% A simple approach is to limit record lengths to smaller than $\lceil \log_2 n\rceil$,
% in order to prohibit an exhaustive enumeration of all records.
% However, there are more space-efficient encodings that require fewer identifiers to be added. 
% For instance, using a base-$(\ell + 1)$ encoding and omitting certain indices
% allows us to encode and recover $(\ell + 1)^\ell - \ell^2$ records with prefixes of length $\ell$.
% It is therefore unclear, what the maximum allowed record length should be.

% Other mitigation strategies that inspect the input directly
% such as limiting the maximum number occurrences per identifier are effective, 
% but burden the victim with additional computation.
% Moreover, they too become complex when trying to guard against various different encodings.
% Limiting the number of occurrences of any identifier specifically 
% comes at the additional risk of accidentally prohibiting honest protocol executions, 
% since certain types of identifiers may naturally occur often.
% IP addresses behind a Network Address Translation (NAT) device serve as an immediate example.

\subsection{Looking towards Deployment}