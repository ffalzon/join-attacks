\section{Analysis of Meta's MKPM Protocol}

In this section, we describe three powerful attacks that exploit the ``leaky'' functionality $\calF_{\LMKPM}$ and enable an adversary to produce a maximum reconstruction of the victim's set in as little as 1 query.
can lead to significantly stronger and more efficient attacks.
Recall that the authors of MK-PrivateID \cite{MKPMC} acknowledge the existence of their protocol's leakage
-- or at least the existence of an informal variant thereof --
but deem it acceptable, as it is an aggregate metric that would only cause concern when involved in multiple executions.
We contradict this assessment by presenting three attacks that achieve Maximum Set Reconstruction
with only a single protocol invocation. 
As before, we assume an input-malicious party $\hat{P}_1$ which interacts with the ``leaky'' MKPM functionality $\LMKPM$. 

% Our three attacks follow a similar paradigm: they involve preprocessing the target set $T$ in such a way that its modified records  are uniquely identifiable in the protocol leakage, 
% which enables the translation of hidden identifiers from the leakage to original plaintext identifiers.
% Exploiting the full leakage pattern then allows us to infer a maximum reconstruction of the victim set \setV{}.
% We start by presenting a baseline attack which is very efficient 
% but makes strong assumptions on the shape of the protocol inputs.
% We then present a three more attacks that gradually weaken said assumptions.
% All attacks presented assume the adversary obtains the full protocol leakage.

% As in the previous chapter, the attacks we present in this section only rely on the leakage pattern,
% but not on the other outputs of \LMKPM{}. For a cleaner presentation of the attacks and correctness proofs,
% we introduce the following notation to denote the leakage stemming from an execution of MK-PrivateID:
% $\leak(\setX, \setY) := (\leakX, \leakY)$, where $(\UID, \MC, (\leakX, \leakY)) \sample \LMKPM(\setX, \setY)$.

% \section{Baseline Attack}
% \label{sec:toy_attack}
% We start with presenting our baseline attack, which will serve as a comparison for further attacks.
% The attack consists of transforming the target set \setT{} into a format which
% -- after invoking the MK-PrivateID protocol and observing the leakage -- 
% will allow the records of the protocol input to be easily matched with the records in the protocol leakage.
% The reader will encounter this paradigm several more times in this chapter.

% Concretely, we make the observation that determining a record matching is trivial
% if there is only one record in \setT. 
% We therefore flatten \setT{} into a modified target set $\setT'$, 
% which only contains one record that holds all identifiers in $\Ispace{\setT}$.
% We then use $\setT'$, together with the leaked $\leak_{\setT'}$,
% as a dictionary to lookup the identifiers corresponding to the hidden identifiers in \leakV.
% An implementation of this attack can be found in \cref{fig:toy_attack}.

% \begin{figure}
% 	\centering
% 	\begin{pchstack}[boxed, center]
% 		\pchspace
% 		\procedure[linenumbering]{$\toyattack^{\LMKPM(\cdot, \setV)}(\setT)$}{%
% 			r \gets () \label{lin:baseline_r_start}\\
% 			\pcfor t \in \setT{} \pcdo \\
% 			\t	\pcfor i = 1, \dots, |t| \pcdo \label{lin:toy_attack_membership_check}\\
% 			\t	\t	\pcif t[i] \not\in r \pcthen \\
% 			\t	\t	\t	r \gets r \ccat t[i] \label{lin:baseline_r_end}\\
% 			\setT' \gets \{r\} \label{lin:toy_attack_def_T_prime} \\
% 			(\UID, \MC, (\leak_{\setT'}, \leakV)) \sample \LMKPM(\setT', \setV) \label{lin:baseline_eval}\\
% 			\{r_e\} \gets \leak_{\setT'} \\
% 			\text{Initialize empty dictionary $D$} \\
% 			\pcfor i = 1, \dots, |r_e| \pcdo \label{lin:baseline_D_start}\\
% 			\t	D[r_e[i]] \gets r[i] \label{lin:baseline_D_end}\\
% 			\recV{\setT'} \gets \dictInfer(\leakV, D) \\
% 			\pcreturn \recV{\setT'}
% 		}
% 		\pchspace
% 		\procedure[linenumbering]{$\dictInfer(\leakV, D)$}{
% 		\text{Initialize empty set $\mathfrak{R}$.} \\
% 		\pcfor v \in \leakV \pcdo \\
% 		\t	r \gets () \\
% 		\t	\pcfor i = 1, \dots, |v| \pcdo \\
% 		\t	\t	\pcif v[i] \in D \pcthen \label{lin:substitute_membership_check} \\
% 		\t	\t	\t	r \gets r \ccat D\left[v[i]\right] \label{lin:substitute_add_id} \\
% 		\t	\mathfrak{R} \gets \mathfrak{R} \cup \{r\} \label{lin:substitute_add_record}\\
% 		\pcreturn \mathfrak{R}
% 	}
% 	\end{pchstack}
% 	\caption{Baseline attack with oracle access to the extended MKPM functionality $\LMKPM(\cdot, \setV)$. 
% 	\setV{} is part of the oracle and is unknown to the adversary.
% 	\dictInfer{} takes a set of hidden records $\leakV$ 
% 	and a dictionary $D:\G \to \Ispace{}$ mapping hidden to plaintext identifiers
% 	and replaces the identifiers in the records of \leakV{} with the corresponding entries in $D$.
% 	Identifiers that do not occur in $D$ are not replaced.} \label{fig:toy_attack} \label{fig:dict_infer}
% \end{figure}

% \begin{lemma}
% 	\label{lem:dictinfer_MR}
% 	Let \setX{} and \setY{} be two sets of records and let $(\leakX, \leakY) \sample \leak(\setX, \setY)$.
% 	Moreover, let $f_C: \Ispace{} \to \G$ be the hiding function sampled in \LMKPM{} to produce \leakX{} and \leakY.
% 	If $D$ is a dictionary mapping $f_C(\id)$ to \id{} for all $\id \in \Ispace{\setX}$,
% 	then $\dictInfer(D, \leakY)$ (\cref{fig:dict_infer}) produces a maximum reconstruction of \setY{} with respect to \setX{}.
% \end{lemma}
% \begin{proof}
% 	Let $\mathfrak{R}$ be the multiset produced by \dictInfer.
% 	Recall from \cref{sec:notation} that we consider different instantiations 
% 	of the same record in $\mathfrak{R}$ as distinct elements.
% 	Note that for each $y \in \leakY$ exactly one reconstructed record $r$
% 	is added to $\mathfrak{R}$.
% 	Let $d: \mathfrak{R} \to \leakY$ denote the function mapping these reconstructed records 
% 	to the corresponding $y$. Clearly, $d$ is a bijection.

% 	We now define an injective function $\varphi: \mathfrak{R} \to \setY$ and prove that it satisfies the three conditions for Maximum Set Reconstruction (\cref{def:max_recons}).
% 	For any record $r \in \mathfrak{R}$ let $(d_1, \dots, d_k)=d(r) \in \leakY$ and let
% 	$$\varphi(r) := (f_C^{-1}(d_1), \dots, f_C^{-1}(d_k))$$
	
% 	Note that $f_C^{-1}(d_j)$ exists for all $r$ and $j$, since $d(r) \in \leakY$ by definition.
% 	Moreover, since $f_C$ is injective, $f_C^{-1}$ is injective for elements in $f_C(\Ispace{})$.
% 	$\varphi$ is injective since $d$ and $f_C^{-1}$ are injective.
% 	We show the three properties from \cref{def:max_recons} separately:
% 	\begin{enumerate}
% 		\item Let $r\in \mathfrak{R}$, $k := |r|$ and $i \leq |r|$. 
% 		Moreover, let $(d_1, \dots, d_{k'}) = d(r)$ for some $k' \geq k$.
% 		By definition of $d$ and by construction of $r$ (line \ref{lin:substitute_add_id}), 
% 		there exists some $j^*\in[k']$ such that $r[i] = D[d_{j^*}]$.
% 		Suppose for a contradiction that $r[i] \not \in \varphi(r)$, that is,
% 		we have $r[i] \neq f_C^{-1}(d_j)$ for all $j \in [k']$.
% 		But since for all $j \in [k']$ with $d_j \in D$ we have $f_C^{-1}(d_j) = D[d_j]$,
% 		we also have that $r[i] \neq D[d_j]$, i.e., there is no $j$ such that $r[i] = D[d_j]$. 
% 		This is a direct contradiction to $r[i] = D[d_{j^*}]$.
% 		% However, if this holds, then the condition on line \ref{lin:substitute_membership_check} 
% 		% would not have been true for any $j\in[k']$ and thus $r[i]$ would not have been added, 
% 		% thus resulting in a contradiction. 
% 		\item Let $r\in \mathfrak{R}$ and $(d_1, \dots, d_{k'}) = d(r)$ for some $k' \in \N$.
% 		Let $\id \in \varphi(r)$, 
% 		i.e., there is some $i \in [k']$ such that $\id = f_C^{-1}(d_i)$.
% 		If $\id \in \Ispace{\setX}$, then by assumption we have $f_C(\id) \in D$ and $D[f_C(\id)] = \id$.
% 		Since $\id = f_C^{-1}(d_i)$, we have $f_C(\id) = f_C(f_C^{-1}(d_i)) = d_i$ 
% 		and thus $d_i \in D$. Therefore, the condition on line~\ref{lin:substitute_membership_check} is met and
% 		$D[d_i] = D[f_C(\id)] = \id$ is added to $r$.
% 		\item Holds trivially, since we have $\varphi(\mathfrak{R}) = \setY$ by the definition of $d$. 
% 	\end{enumerate}
% The lemma thus follows.
% \end{proof}

% \begin{lemma}
% 	For any target set \setT{} and victim set \setV{}, 
% 	$\toyattack^{\LMKPM(\cdot, \setV)}(\setT)$ from \cref{fig:toy_attack}
% 	produces a maximum reconstruction of \setV{} with respect to \setT{}.
% \end{lemma}
% \begin{proof}
% 	It is easy to see that the record $r$ constructed in lines \ref{lin:baseline_r_start}~-~\ref{lin:baseline_r_end} contains all identifiers from 
% 	$\Ispace{\setT}$ and therefore $\Ispace{\setT} = \Ispace{\setT'}$. 
% 	It therefore suffices to show that $\toyattack$ produces a maximum reconstruction of $\setV$ w.r.t. $\setT'$.
	
% 	Let $f_C$ be the identifier hiding function sampled during the invocation of 
% 	\LMKPM{} in line \ref{lin:baseline_eval},  
% 	let $D$ be the dictionary constructed in lines \ref{lin:baseline_D_start} and \ref{lin:baseline_D_end} 
% 	and let $k = |r|$.
% 	By definition of \LMKPM{}, we have $r_e = (f_C(r[1]), \dots, f_C(r[k]))$
% 	and therefore $D[f_C(r[i])] = D[r_e[i]] = r[i]$ for all $i \in [k]$.
% 	Since $r$ contains all identifiers from $\Ispace{\setT'}$, the claim follows from \cref{lem:dictinfer_MR}.
% \end{proof}

% Note that the attack creates a reconstruction for every record in \leakV,
% even if the corresponding record in \setV{} contains no identifiers from \Ispace{\setT}.
% In that case, the reconstructed record simply remains empty. 
% This does not violate \cref{def:max_recons} and simplifies the presentation and correctness proof of the attack.
% Moreover, after carrying out the attack, the adversary can remove these empty records from the reconstruction 
% without affecting its validity as a maximum reconstruction.

% \textbf{Time and Query Complexity.}
% Recall from \cref{sec:notation} that we assume that reading from, writing to, 
% and checking the presence of keys in dictionaries requires constant time.
% Let $\lambda$ denote the maximum record length in \setT{} and \setV{} and let $n := |\setT|$ and $m := |\setV|$.
% The attack has an asymptotic time complexity of $\bigO{(n\lambda)^2 + \lambda m}$.
% The membership check in line \ref{lin:toy_attack_membership_check} could be reduced to constant time by maintaining 
% a set of already added identifiers, since membership checks in the latter can be done in constant time.
% This reduces to time complexity to \bigO{\lambda(n+m)} at the cost of increasing the space consumption.
% This increase in space consumption could be avoided using approximate membership query data structures such as Bloom Filters.
% However, these data structures usually yield false positive answers with a certain probability, 
% which may lead to an incomplete reconstruction.  
% The attack only requires one protocol invocation.

% \textbf{Discussion.}
% The attack sets an upper bound on the strength of the assumptions
% necessary for the attack to work, which we will reduce substantially throughout the remainder of this chapter. 
% Concretely, the single record contained in $\setT'$ has length $|\Ispace{\setT}|$.
% The attack therefore assumes that the length of records may be arbitrary, in particular, may be linear in the input size.
% While not excluded by neither the MK-PrivateID specification nor its implementation,
% this is easily detectable by the victim, as it learns the format of the adversary's input in its own leakage $\leak^P_{\setT'}$ (see \cref{fig:MKPM}).
% Aborting the protocol execution in such a case provides an easy mitigation.

% \subsection{A First Optimization}
% One can circumvent the need for arbitrarily long records, 
% with the simple observation that the lengths of records are preserved in the protocol leakage.
% This insight allows splitting the single long record in $\setT'$
% into multiple smaller records with unique lengths.
% Call this new set of records $\setT''$. 
% A record matching between $\setT''$ and $\leak_{\setT''}$ can then be inferred based on
% record lengths. 
% This optimized attack, too, requires only one protocol invocation and has a time complexity of \bigO{(n\lambda)^2 + \lambda m}, 
% which could be reduced to \bigO{\lambda(n+m)} with optimizing membership checks, as mentioned above.
% However, the maximum record length has decreased from $|\Ispace{\setT}|$ to $\frac{\sqrt{1+8|\Ispace{\setT}|}-1}{2} \approx \sqrt{2|\Ispace{\setT}|}$.

% While (arguably) less conspicuous to the naked eye, 
% the victim can still detect such an attack by checking whether the adversary's input contains no two records with the same length.
% As before, the victim can abort the protocol execution in this case to prevent the attack.
% A much simpler mitigation, however, 
% is to enforce all records to have the same length and abort protocol execution otherwise.
% This does not restrict the protocol's deployment, as shorter records can simply be padded with random bit strings.
% While the authors of~\cite{MPMC} also suggest this measure to prevent leaking the record lengths,
% they did not include it in their implementation of MK-PrivateID.

% \section{Record Enumeration Attack}
% In light of the aforementioned mitigation, we propose a more general attack,
% whose inputs are records of equal lengths.
% In particular, we make the assumption that all records in the target set $\setT$
% contain $\lambda$ identifiers for some $\lambda\in\N$.
% Crucially, the attack we present in this section is successful if this assumption is met, 
% but does not rely on it. This sets it apart from the previous attack, 
% which only succeeds if the record lengths are unique and preserved by the protocol leakage. 
% Instead, our next attack only leverages the fact that \LMKPM{} preserves the order of identifiers in records of \setT{} and
% hides identifiers deterministically, 
% i.e. two occurrences of the same identifier result in the same hidden identifier in the same protocol run.

% \begin{figure}
% 	\centering
% 	{\def\rowspace{-.4pt}
% 	\def\minwidthemail{1.95cm}
% 	\def\minwidthphone{2.1cm}
% 	\def\minwidthid{.6cm}
% 	\def\linecorr{-.4pt}
% 	\def\blockminheight{1.3em}
% 	\definecolor{added-id}{HTML}{15b01a}
% 	\definecolor{matched-light}{HTML}{DCEDC8}
% 	\definecolor{matched-dark}{HTML}{C4E1A3}
% 	\definecolor{tiebreaker}{HTML}{A3CEE1}


% 	\begin{subfigure}{.5\textwidth}
% 		\centering
% 		\begin{tikzpicture}
  [block/.style={draw,minimum width=#1, minimum height=\blockminheight,font={\small}},
	block/.default=10em,high/.style={},auto,
	node distance=0, 
	>=Stealth]

% Nodes
\node[block=\minwidthid, draw=none ] (alice_idx) {$\bx_0:$};
\node[block=\minwidthid,fill=matched-light, right=\linecorr of alice_idx] (alice1) {$v_0$};
\node[block=\minwidthid,fill=matched-light, right=\linecorr of alice1] (alice2) {$v_0$};
\node[block=\minwidthemail,fill=trow-dark, right=\linecorr of alice2] (alice3) {alice@mail.com};
\node[block=\minwidthphone,fill=trow-dark, right=\linecorr of alice3] (alice4) {+41 98 765 43 21};

\node[block=\minwidthid, draw=none, below=\rowspace of alice_idx] (bob_idx) {$\bx_1:$};
\node[block=\minwidthid,fill=trow-light, right=\linecorr of bob_idx] (bob1) {$v_0$};
\node[block=\minwidthid,fill=trow-light, right=\linecorr of bob1] (bob2) {$v_1$};
\node[block=\minwidthemail,fill=trow-light, right=\linecorr of bob2] (bob3) {bob@mail.com};
\node[block=\minwidthphone,fill=trow-light, right=\linecorr of bob3] (bob4) {+41 12 345 67 89};

\node[block=\minwidthid,below=\rowspace of bob_idx, draw=none ] (eve_idx) {$\bx_2:$};
\node[block=\minwidthid,fill=trow-dark, right=\linecorr of eve_idx] (eve1) {$v_1$};
\node[block=\minwidthid,fill=trow-dark, right=\linecorr of eve1] (eve2) {$v_0$};
\node[block=\minwidthemail,fill=trow-dark, right=\linecorr of eve2] (eve3) {eve@mail.com};
\node[block=\minwidthphone,fill=trow-dark, right=\linecorr of eve3] (eve4) {+41 45 678 91 23};

\node[block=\minwidthid,below=\rowspace of eve_idx, draw=none ] (gus_idx) {$\bx_3:$};
\node[block=\minwidthid,fill=trow-light, right=\linecorr of gus_idx] (gus1) {$v_1$};
\node[block=\minwidthid,fill=tiebreaker, right=\linecorr of gus1] (gus2) {$v_2$};
\node[block=\minwidthemail,fill=trow-light, right=\linecorr of gus2] (gus3) {gus@mail.com};
\node[block=\minwidthphone,fill=trow-light, right=\linecorr of gus3] (gus4) {+41 56 789 12 34};

% Rectangles
\draw[draw=added-id, line width=.7pt] (alice1.north west) rectangle (gus2.south east); % green
\draw[draw=highlight-blue, line width=.7pt] (alice3.north west) rectangle (gus4.south east); % blue

\end{tikzpicture}

% 		\caption{$\setT'$}\label{subfig:recenum_example_plain}
% 	\end{subfigure}
% 	\hspace{1cm}
% 	\begin{subfigure}{.25\textwidth}
% 		\centering
% 		
\begin{tikzpicture}
	[block/.style={draw,minimum width=#1, minimum height=\blockminheight,font={\small}},
	block/.default=10em,high/.style={},auto,
	node distance=0, 
	>=Stealth]
	% Nodes
	\node[block=\minwidthid, draw=none] (eve_idx) {$\bx_2:$};
	\node[block=\minwidthid,fill=trow-dark, right=\linecorr of eve_idx](eve1) {$e_1$};
	\node[block=\minwidthid,fill=trow-dark, right=\linecorr of eve1] (eve2) {$e_2$};
	\node[block=\minwidthid,fill=trow-dark, right=\linecorr of eve2] (eve3) {$e_3$};
	\node[block=\minwidthid,fill=trow-dark, right=\linecorr of eve3] (eve4) {$e_4$};

	\node[block=\minwidthid,below=\rowspace of eve_idx, draw=none ] (bob_idx) {$\bx_1:$};
	\node[block=\minwidthid,fill=trow-light, right=\linecorr of bob_idx] (bob1) {$e_2$};
	\node[block=\minwidthid,fill=trow-light, right=\linecorr of bob1] (bob2) {$e_1$};
	\node[block=\minwidthid,fill=trow-light, right=\linecorr of bob2] (bob3) {$e_5$};
	\node[block=\minwidthid,fill=trow-light, right=\linecorr of bob3] (bob4) {$e_6$};

	\node[block=\minwidthid,below=\rowspace of bob_idx, draw=none ] (gus_idx) {$\bx_3:$};
	\node[block=\minwidthid,fill=trow-dark, right=\linecorr of gus_idx] (gus1) {$e_1$};
	\node[block=\minwidthid,fill=tiebreaker, right=\linecorr of gus1] (gus2) {$e_7$};
	\node[block=\minwidthid,fill=trow-dark, right=\linecorr of gus2] (gus3) {$e_8$};
	\node[block=\minwidthid,fill=trow-dark, right=\linecorr of gus3] (gus4) {$e_9$};
	
	\node[block=\minwidthid,below=\rowspace of gus_idx, draw=none ] (alice_idx) {$\bx_0:$};
	\node[block=\minwidthid,fill=matched-light, right=\linecorr of alice_idx] (alice1) {$e_2$};
	\node[block=\minwidthid,fill=matched-light, right=\linecorr of alice1] (alice2) {$e_2$};
	\node[block=\minwidthid,fill=trow-light, right=\linecorr of alice2] (alice3) {$e_{10}$};
	\node[block=\minwidthid,fill=trow-light, right=\linecorr of alice3] (alice4) {$e_{11}$};

	\draw[draw=added-id, line width=.7pt] (eve1.north west) rectangle (alice2.south east);
\end{tikzpicture}
% 		\caption{$\leak_{\setT'}$}\label{subfig:recenum_example_leakage}
% 	\end{subfigure}}
% 	\caption{Example of a target set transformation performed by \recenumattack{}. 
% 	(\subref{subfig:recenum_example_plain}) shows the preprocessed set $\setT'$. 
% 	The prepended sequences encoding the set enumeration have a green border.
% 	Note the third identifier $\id_2$ in the last record. 
% 	(\subref{subfig:recenum_example_leakage}) shows the (partial) protocol leakage $\leak_{\setT'}$.
% 	The encoding of zero is highlighted in green. The hidden $\id_2$ is highlighted in blue.}
% 	\label{fig:recenum_example}
% \end{figure}

% We leverage these properties by enumerating the records in the target set \setT{} and
% prepending an encoding of their respective index as a sequence of known identifiers.
% If this enumeration can be recovered in the protocol leakage, 
% we can establish a mapping from hidden to plaintext identifiers and, 
% thus, use \dictInfer{} (see \cref{fig:dict_infer}) to infer a maximum reconstruction of the victim set \setV{}.

% Concretely, let $\id_0$ and $\id_1$ be two arbitrary but distinct identifiers.
% Whether they already occur in the target or victim sets does not matter.
% We pick some order $(t_0, \dots, t_{n-1})$ among the records of \setT{}.
% Some index $i\in \{0, \dots, n-1\}$ can be encoded as a sequence of $\id_0$ and $\id_1$
% by taking the binary representation of $i$ and replacing all zeros with $\id_0$ and all ones with $\id_1$ respectively.
% This sequence has length $\ell := \lceil \log_2 n\rceil$ for $n = |\setT|$ and 
% is prepended to the record $t_i$. We denote the set of these extended records by $\setT'$. 
% See \cref{subfig:recenum_example_plain} for a small example.
% The encoding procedure \IdxToSeq{} is depicted in \cref{fig:idx_to_seq}.

% We now consider the leakage that results from invoking MK-PrivateID on $\setT'$;
% let $(\leak_{\setT'}, \leakV) \sample \leak(\setT', \setV)$.
% Recall that \LMKPM{} hides identifiers using an injective random function $f_C$.
% Since \LMKPM{} preserves the order of the identifiers in records of $\setT'$,
% the first $\ell$ identifiers of any $t^* \in \leak_{\setT'}$ therefore 
% only contains two (hidden) identifiers $e$ and $e'$.
% Moreover, we either have $e=f_C(\id_0)$ and $e'=f_C(\id_1)$ or $e=f_C(\id_1)$ and $e'=f_C(\id_0)$.
% By determining which of the two cases applies, one can recover the index $i$ encoded in the prefix 
% of $t^*$ and establish a mapping between the hidden identifiers of $t^*$ and the plain identifiers 
% of its corresponding record $t_i \in \setT'$.
% See procedure \SeqToIdx{} in \cref{fig:seq_to_idx} for details.

% \begin{figure}
% 	\pchspace
% 	\begin{pchstack}[boxed, center]
% 	\begin{pcvstack}
% 	\procedure[linenumbering]{$\recenumattack^{\LMKPM(\cdot, \setV)}(\setT)$}{
% 		\text{Pick three distinct identifiers $\id_0, \id_1, \id_2$.} \\
% 		\text{Pick some ordering $\setT = (t_1, \dots, t_n)$.} \\
% 		\ell \gets \lceil \log_2 n \rceil \\
% 		\pcfor i = 1, \dots, n \pcdo\\
% 		\t	s \gets \IdxToSeq(i, \ell, \id_0, \id_1, \id_2) \\
% 		\t	t'_i \gets s \ccat t_i \label{lin:recenum_t_prime}\\
% 		(\UID, \MC, (\leak_{\setT'}, \leakV)) \sample \LMKPM(\{t'_1, \dots, t'_n\}, \setV) \label{lin:recenum_evaluation} \pcskipln\\
% 		\pclinecomment{Find $f_C(\id_0)$} \\
% 		\text{Find $t^*_0 \in \leak_{\setT'}$ s.t. $t^*_0[1] = \dots = t^*[\ell]$} \label{lin:recenum_t0}\\
% 		e_0 \gets t^*_0[1] \label{lin:recenum_e0} \label{lin:assign_e0} \pcskipln\\
% 		\pclinecomment{Construct substitution map $D$. }\\
% 		\text{Initialize empty dictionary $D$.} \\
% 		\pcfor t^* \in \leak_{\setT'} \pcdo \label{lin:recenum_D_start} \\
% 		\t	i \gets \SeqToIdx(t^*, \ell, e_0) \\
% 		\t	\pcfor j = 1, \dots, |t^*| \pcdo \\
% 		\t	\t	D\left[t^*[j]\right] \gets t'_i[j] \label{lin:recenum_D_end}\\
% 		\recVT \gets \dictInfer(\leakV, D) \\
% 		\pcreturn \recVT
% 	}
% 	\end{pcvstack}
% 	\pchspace
% 	\begin{pcvstack}
% 	\procedure[linenumbering, lnstart=18]{$\IdxToSeq(i, \ell, \id_0, \id_1, \id_2)$}{
% 		\pcif i = 2^\ell - 1 \pcthen \\
% 		\t	\pcreturn (\id_1)^{\ell - 1} \ccat \id_2 \\
% 		s \gets () \\
% 		b \gets \binary{\ell}(i)\\
% 		\pcfor j = 1,\dots, \ell \pcdo \label{lin:idx_to_seq_bin_encode_start}\\
% 		\t	\pcif b_j = 0 \pcthen \\
% 		\t	\t 	s \gets s \ccat \id_0 \\
% 		\t	\pcelse \\
% 		\t	\t	s \gets s \ccat \id_1 \label{lin:idx_to_seq_bin_encode_end} \\
% 		\pcreturn s
% 	}
% 	\procedure[linenumbering, lnstart=28]{$\SeqToIdx(t^*, \ell, e_0)$}{
% 		\pcif \forall j \in [\ell] : t^*[j] \neq e_0 \pcthen \label{lin:seq_to_idx_tiebreak} \pcskipln\\
% 		\t	\pclinecomment{Encoding is $(\id_1, \dots, \id_1, \id_2)$} \\
% 		\t	\pcreturn 2^\ell - 1\\
% 		i \gets 0 \\
% 		\pcfor j = 1, \dots, \ell \pcdo \\
% 		\t	\pcif s[j] \neq e_0 \pcthen \\
% 		\t	\t	i \gets i + 2^{\ell - i} \\
% 		\pcreturn i
% 	}
% 	\end{pcvstack}
% 	\end{pchstack}
% 	\caption{Record enumeration attack \recenumattack{} with oracle access to the extended MK-PrivateID functionality $\LMKPM(\cdot, \setV)$. 
% 	Note that \setV{} is part of the oracle and not known to the adversary.

% 	\dictInfer{} is defined in \cref{fig:toy_attack}.}\label{fig:rec_enum_attack}\label{fig:idx_to_seq}\label{fig:seq_to_idx}
% \end{figure}

% To decide whether $e$ corresponds to $\id_0$ or $\id_1$, assume that $n$ is not a power of two,
% i.e., there are less than $2^\ell$ records in \setT{}.
% Thus, no record will be assigned index $2^\ell -1$ and the corresponding encoding $(\id_1, \dots, \id_1)$
% will not occur. This acts as a tiebreaker, since the encoding of index zero, $(\id_0, \dots, \id_0)$, clearly does occur.
% We can identify the record containing said encoding in the leakage, as it is the only record in $\leak_{\setT'}$
% whose first $\ell$ identifiers coincide. We call this record $t^*_0$ and clearly, we have $id_0 = f_C(t_0^*[1])$.
% Continuing the aforementioned example, \cref{subfig:recenum_example_leakage} shows a possible protocol leakage with the previously enumerated set.
% The full attack is displayed in \cref{fig:rec_enum_attack}.

% The assumption that $n$ is not a power of two incurs no loss of generality.
% Should this not be the case, the adversary can simply add or remove one record from the target set.
% If this is not an option, the adversary can also
% introduce a third, tie-breaking identifier $\id_2 \not\in\{\id_0, \id_1\}$ and replace the last identifier in the encoding 
% $(\id_1, \dots, \id_1)$ with $\id_2$. 
% This is the approach we take in our presentation of the attack.

% \begin{theorem}
% 	Let \setT{} and \setV{} be two sets of records.
% 	$\recenumattack^{\LMKPM(\cdot, \setV)}(\setT)$ (\cref{fig:rec_enum_attack}) produces a maximum reconstruction of \setV{} w.r.t. \setT{} (\cref{def:max_recons}).
% \end{theorem}

% \begin{proof}
% In order to prove the theorem we prove the following proposition. 
% The theorem follows directly from Proposition~\ref{thm:recenum_D} and \cref{lem:dictinfer_MR}.
% \begin{proposition}\label{thm:recenum_D}
% 	Let \setT{} and \setV{} be two sets of records and let
% 	$D$ be the dictionary constructed during an execution of $\recenumattack^{\LMKPM(\cdot, \setV)}(\setT)$ (\cref{fig:rec_enum_attack}).
% 	Moreover, let $f_C$ be the identifier hiding function sampled during the evaluation of \LMKPM{} in line~\ref{lin:recenum_evaluation} in \cref{fig:rec_enum_attack}. 
% 	We have $D[f_C(\id)] = \id$ for all $\id \in \Ispace{\setT}.$
% \end{proposition}
% \begin{proof} \textbf{of Proposition~\ref{thm:recenum_D}} 
% 	Let $\setT' = \{t'_0, \dots, t'_{n-1}\}$, where $t'_i$ denotes the altered target record constructed in 
% 	line~\ref{lin:recenum_t_prime} for $i \in \{0, \dots, n - 1\}$.
% 	We show that $D[f_C(\id)] = \id$ for all $\id \in \setT'$.
% 	Since $\Ispace{\setT} \subseteq \Ispace{\setT'}$, this implies the claim.
	
% 	We first show that $e_0 = f_C(\id_0)$ for the value $e_0$ defined on line~\ref{lin:assign_e0}.
% 	Note that the transformation of an index's binary representation to a sequence of $\id_0$ and $\id_1$
% 	in lines \ref{lin:idx_to_seq_bin_encode_start}~-~\ref{lin:idx_to_seq_bin_encode_end} of \IdxToSeq{}
% 	could produce two encodings that only contain one identifier, namely, $(\id_0)^\ell$ and $(\id_1)^\ell$
% 	for $i = 0$ and $i = 2^\ell - 1$ respectively.
% 	However, the latter is prevented by the condition on line~\ref{lin:seq_to_idx_tiebreak} and therefore
% 	\IdxToSeq{} only outputs one sequence where all identifiers are equal.
% 	Hence, there is only one record in $\setT'$ whose first $\ell$ identifiers all coincide,
% 	namely, $t'_0$ with encoding $(\id_0)^\ell$.
% 	Since $f_C$ is injective and the identifiers in records of $\setT'$ are not shuffled, 
% 	the latter also holds for $\leak_{\setT'}$.
% 	Therefore, $t^*_0 = (f_C(t'_0[1]), \dots, f_C(t'_0[\lambda + \ell]))$ (line~\ref{lin:recenum_t0}).
% 	Since $t'_0[1] = \id_0$, we have $e_0 = t^*_0[1] = f_C(t'_0[1]) = f_C(\id_0)$ (see line~\ref{lin:recenum_e0}).
	
% 	We now show that the attack correctly recovers indices from $\leak_{\setT'}$. 
% 	We have already established that $t^*_0$ is correctly matched with $t'_0$,
% 	as it's encoding is the only one containing only $\id_0$. 
% 	Any other encoding $s$ produced by \IdxToSeq{}
% 	either (a) is $(\id_1)^{\ell - 1} \ccat \id_2$ or (b) only
% 	contains the identifiers $\id_0$ and $\id_1$.
% 	\begin{description}
% 	\item[Case (a):] This case is only produced on input $i = 2^\ell - 1$ and the resulting vector of hidden identifiers in the 	
% 		 leakage is $(f_C(\id_1))^{\ell - 1} \ccat f_C(\id_2)$. 
% 		 This satisfies the condition on line~\ref{lin:seq_to_idx_tiebreak} of \SeqToIdx{}, 
% 		 which therefore yields the correct index $2^\ell - 1$.
% 	\item[Case (b):] In this case, the vector of hidden identifiers corresponding to $s$ in the leakage, which we denote
% 		$s^*:=(f_C(s[1]),\allowbreak\dots,f_C(s[\ell]))$, does not satisfy the condition on line~\ref{lin:seq_to_idx_tiebreak}, 
% 		since the only encoding that would do so is $(\id_1)^{\ell}$. 
% 		But this is never output by \IdxToSeq{}. 
% 		Therefore, $s^*$ is interpreted as a binary encoding, where every occurrence of $e_0$ 
% 		is interpreted as $0$ and every other identifier as $1$.
% 		The correctness of this recovery therefore follows from the fact that $e_0 = f_C(\id_0)$,
% 		which we established above.
% 	\end{description}
	
% 	By the definition of \LMKPM{}, we have $t^*_i = (f_C(t'_i[1]), \dots, f_C(t'_i[\lambda+\ell]))$ for all $i \in \{0, \dots, n-1\}$.
% 	Since $\Ispace{\setT'} := \bigcup_{\substack{l}{t\in\setT'\\\id \in t}} \id$, 
% 	the claim holds by the construction of $D$ in lines~\ref{lin:recenum_D_start}~-~\ref{lin:recenum_D_end}.
% \end{proof}
	
% % \begin{proof}
% 	The theorem follows directly from \cref{thm:recenum_D} and \cref{lem:dictinfer_MR}.
% \end{proof}


% \textbf{Time and Query Complexity.}
% Assuming $\binary{\ell}(i)$ can be determined in time $\bigO{\ell}=\bigO{\log n}$, 
% \IdxToSeq{} requires time $\bigO{\ell}$.
% The same holds for \SeqToIdx{}.
% Encoding the target set therefore requires time \bigO{n\log_2 n}.
% Finding $t^*_0$ requires iterating over $\leak_{\setT'}$ and inspecting the first $\ell$ identifiers of each record,
% incurring an effort of \bigO{n\log_2 n} as well.
% The construction of $D$ requires \bigO{n (\log_2 n + \lambda)} time
% and \dictInfer{} \bigO{\lambda m}, as previously established.
% This results in an overall time complexity of \bigO{n (\log_2 n + \lambda) + \lambda m}.
% Like the baseline attack, \recenumattack{} requires only a single protocol invocation.

% \textbf{Discussion.}
% Compared to the baseline attack, the records are now much shorter and are of uniform length.
% Specifically, records are now of length $\lceil \log_2 n\rceil + \lambda$,
% where $\lambda$ denotes the original record length in the target set.
% Ignoring the constant $\lambda$, this is an exponential reduction from the maximum record lengths 
% $|\Ispace{\setT}|$ and $\approx \sqrt{2|\Ispace{\setT}|}$
% for the baseline attack and its first optimization respectively.
% At the same time, \recenumattack{} makes considerably weaker assumptions on the set of admissible inputs, thus,
% resulting in a clear improvement.

% The mitigation of the record enumeration attack is clearly more 
% involved than with the baseline attack,
% since MK-PrivateID heavily relies on the preservation of the identifier order 
% and the deterministic hiding of identifiers
% in order to compute its multi-key matching logic (see \cref{fig:match_logic}).
% A simple approach is to limit record lengths to smaller than $\lceil \log_2 n\rceil$,
% in order to prohibit an exhaustive enumeration of all records.
% However, there are more space-efficient encodings that require fewer identifiers to be added. 
% For instance, using a base-$(\ell + 1)$ encoding and omitting certain indices
% allows us to encode and recover $(\ell + 1)^\ell - \ell^2$ records with prefixes of length $\ell$.
% It is therefore unclear, what the maximum allowed record length should be.

% Other mitigation strategies that inspect the input directly
% such as limiting the maximum number occurrences per identifier are effective, 
% but burden the victim with additional computation.
% Moreover, they too become complex when trying to guard against various different encodings.
% Limiting the number of occurrences of any identifier specifically 
% comes at the additional risk of accidentally prohibiting honest protocol executions, 
% since certain types of identifiers may naturally occur often.
% IP addresses behind a Network Address Translation (NAT) device serve as an immediate example.

% \section{Snake Attack}
% \label{sec:snake_attack}
% \begin{figure}
% 	\centering
% 	{\def\rowspace{.4cm}
% 	\def\minwidthemail{2cm}
% 	\def\minwidthphone{2.1cm}
% 	\def\minwidthid{.6cm}
% 	\def\linecorr{-.4pt}
% 	\def\blockminheight{1.3em}
% 	\definecolor{added-id}{HTML}{15b01a}
% 	\begin{subfigure}[t][][c]{0.48\textwidth}
% 		\centering
% 		\begin{tikzpicture}
	[block/.style={
    draw,
    minimum width=#1,
    minimum height=\blockminheight,
    font=\small,
    text height=1.2ex,
    text depth=.25ex
  	}
	]
	%node distance=5em,auto]
	% Nodes
	\node[block=\minwidthemail,fill=trow-dark] (alice1) {alice@mail.com};
	\node[block=\minwidthphone,fill=trow-dark, right=\linecorr of alice1] (alice2) {+41 98 765 43 21};
	\node[block=\minwidthemail,fill=trow-dark, right=\linecorr of alice2, draw=added-id] (alice3) {bob@mail.com};

	\node[block=\minwidthemail,fill=trow-light, below=\rowspace of alice1] (bob1) {bob@mail.com};
	\node[block=\minwidthphone,fill=trow-light, right=\linecorr of bob1] (bob2) {+41 12 345 67 89};
	\node[block=\minwidthemail,fill=trow-light, right=\linecorr of bob2, draw=added-id] (bob3) {eve@mail.com};

	\node[block=\minwidthemail,fill=trow-dark, below=\rowspace of bob1] (eve1) {eve@mail.com};
	\node[block=\minwidthphone,fill=trow-dark, right=\linecorr of eve1] (eve2) {+41 45 678 91 23};
	\node[block=\minwidthemail,fill=trow-dark, right=\linecorr of eve2, draw=added-id] (eve3) {$v$};
	% Connections
	\draw[->] (alice3) to[out=225, in=60, looseness=.5] (bob1);
	\draw[->] (bob3) to[out=225, in=60, looseness=.5] (eve1);
\end{tikzpicture}


% 		\caption{}
% 		\label{subfig:snake_example_unique}
% 	\end{subfigure}
% 	\begin{subfigure}[t][][c]{0.48\textwidth}
% 		\centering
% 		\begin{tikzpicture}
	[block/.style={
    draw,
    minimum width=#1,
    minimum height=\blockminheight,
    font=\small,
    text height=1.2ex,
    text depth=.25ex
  	}
	]
	% Nodes
	\node[block=\minwidthid,fill=trow-dark, draw=added-id] (alice0) {$\id_0$};
	\node[block=\minwidthemail,fill=trow-dark, right=\linecorr of alice0, behind path] (alice1) {alice@mail.com};
	\node[block=\minwidthphone,fill=trow-dark, right=\linecorr of alice1] (alice2) {+41 98 765 43 21};
	\node[block=\minwidthid,fill=trow-dark, right=\linecorr of alice2, draw=added-id] (alice3) {$\id_1$};

	\node[block=\minwidthid,fill=trow-light, below=\rowspace of alice0, draw=added-id] (bob0) {$\id_1$};
	\node[block=\minwidthemail,fill=trow-light, right=\linecorr of bob0, behind path] (bob1) {bob@mail.com};
	\node[block=\minwidthphone,fill=trow-light, right=\linecorr of bob1] (bob2) {+41 12 345 67 89};
	\node[block=\minwidthid,fill=trow-light, right=\linecorr of bob2, draw=added-id] (bob3) {$\id_2$};

	\node[block=\minwidthid,fill=trow-dark, below=\rowspace of bob0, draw=added-id] (eve0) {$\id_2$};
	\node[block=\minwidthemail,fill=trow-dark, right=\linecorr of eve0, behind path] (eve1) {alice@mail.com};
	\node[block=\minwidthphone,fill=trow-dark, right=\linecorr of eve1] (eve2) {+41 12 345 67 89};
	\node[block=\minwidthid,fill=trow-dark, right=\linecorr of eve2, draw=added-id] (eve3) {$\id$};
	% Connections
	\draw[->] (alice3) to[out=240, in=60, looseness=.3] (bob0);
	\draw[->] (bob3) to[out=240, in=60, looseness=.3] (eve0);

	% Draw boxes over first row again to cover neighboring border
	\node at (alice0) [block=\minwidthid, draw=added-id]{};
	\node at (bob0) [block=\minwidthid, draw=added-id]{};
	\node at (eve0) [block=\minwidthid, draw=added-id]{};

	% \draw[->] (n1) -- (n2);
	% \draw[->] (n2) -- (n3);
	% \draw[->] (n3) -- (n4);
	% \coordinate (x) at ([yshift=-1cm]n3.south);
	% \coordinate (y) at (start |- x);
	% \draw (start) -- (y) (x) edge[->] (n3.south) (x) edge["EXAMPLE"] (y);
	% {[on background layer]
    % \fill[red!30] (alice1.north west) rectangle (alice3.south east); 
    % }
\end{tikzpicture}
% 		\caption{}
% 		\label{subfig:snake_example_general}
% 	\end{subfigure}
% 	}
% 	\caption{Results of preprocessing a set of records \textbf{(\subref{subfig:snake_example_unique})} if there is a column that only contains unique identifiers 
% 	and \textbf{(\subref{subfig:snake_example_general})} if an artificial unique column needs to be added.
% 	The original records have black borders. Added identifiers are highlighted in green. Arrows indicate the (implicit) linked list.}\label{fig:snake_exmp}
% \end{figure}
% While the record enumeration attack marks a significant improvement over the baseline attack in terms of detectability,
% the lengths of the modified records still depend on the input size.
% This is impractical, especially in scenarios with large inputs.
% In this section, we overcome this limitation with the observation that
% an order can be imposed on a set of records without explicitly assigning 
% a unique index to each record.

% Assume, for now, that there exists a column $j^*$ in \setT{} in which every identifier is unique, i.e., $|\{t[j^*] \setdsc t \in \setT\}| = |\setT|$.
% This is not an unreasonable assumption, since certain identifiers such as email addresses are usually unique across user data.
% Should no such column exist, the adversary can simply create one by selecting $n$ distinct identifiers
% and prepending one of them to each record in \setT{}, in which case $j^* = 1$.

% We can then link the records of \setT{} by first choosing some arbitrary ordering $\setT = \{t_1, \dots, t_n\}$ 
% and appending the unique index of the $i+1$-th record to the $i$-th record for all $i \in [n-1]$.
% See \cref{subfig:snake_example_unique} for a small example. The linking procedure \snakeencodeUnique{} 
% is implemented in \cref{fig:snakeencode1}.
% The procedure outputs the modified target set $\setT'$ as a list $(t'_1, \dots, t'_n)$,
% such that $t'_i$ is located at the $i$-th position in the (implicit) linked list.

% \begin{figure*}
% 	\centering
% 	\Wider[2em]{
% 	\begin{pchstack}[boxed, center]
% 		\begin{pcvstack}
% 			\procedure[linenumbering]{$\snakeencodeUnique(j^*, (t_1, \dots, t_n))$}{
% 				\text{Initialize empty dictionary $E$.} \\
% 				\pcfor i = 1, \dots, n - 1 \pcdo \\
% 				\t	t'_i \gets t_i \ccat t_{i+1}[j^*] \\
% 				\text{Pick some $\id \not \in \{t_i[j^*] \setdsc i \in [n]\}$}  \label{lin:snake_encode_last_id}\\
% 				t'_n \gets t_n \ccat \id \\
% 				\pcreturn (t'_1, \dots, t'_n)
% 			}
% 			% \procedure[linenumbering]{$\snakeencodeGeneral(\setT)$}{
% 			% 	\text{Initialize empty dictionary $E$.} \\
% 			% 	\text{Fix some ordering $\setT = \{t_1, \dots, t_n\}$.} \\
% 			% 	\text{Pick $n+1$ distinct identifiers $\id_0, \dots \id_n \in \Ispace{}$.} \\
% 			% 	\pcfor i = 1 \dots, n \pcdo \\
% 			% 	\t	t'_i \gets \id_i \ccat t_i \ccat id_{i+1} \\
% 			% 	\t	E[i] \gets t' \\
% 			% 	\pcreturn (t'_1, \dots, t'_n)
% 			% }
% 		\end{pcvstack}
% 		\pchspace
% 		\begin{pcvstack}
% 			\procedure[linenumbering]{$\snakerecoverUnique(\leak_{\setT'}, j^*)$}{
% 				\text{Initialize empty dictionary $\loc$.}\\
% 				\text{Fix some ordering $\leak_{\setT'} = (h_1, \dots, h_n)$.} \\
% 				k' \gets |h_1| \pccomment{$= |h_2| = \dots = |h_n|$} \pcskipln\\
% 				\pclinecomment{Store locations of linked identifiers}\\
% 				\pcfor i = 1, \dots, n \pcdo \\
% 				\t	\loc[h_i[j^*]] \gets i \pcskipln\\
% 				\pclinecomment{Find start index $c$ of the linked list} \\
% 				\text{Find $c \leq n$ s.t. $h_c[j^*]\not\in\{h_i[k'] \setdsc i \in [n]\}$} \label{lin:snake_recover_c} \pcskipln\\
% 				\pclinecomment{Traverse linked list to recover original ordering}\\
% 				i \gets 1 \\
% 				\pcdo \label{lin:snake_recover_dowhile}\\
% 				\t	t^*_i := h_{c} \label{lin:snake_recover_assign_t_star}\\
% 				\t	c \gets \loc[h_c[k']] \label{lin:snake_recover_updt_c}\\
% 				\t	i \gets i + 1\\
% 				\pcwhile h_{c}[k'] \in \loc \label{lin:snake_recover_dowhile_end}\\
% 				\pcreturn (t^*_1, \dots, t^*_n)
% 			}
% 		\end{pcvstack}
% 	\end{pchstack}}
% 	\caption{Procedures \snakeencodeUnique{} and \snakerecoverUnique{}.}\label{fig:snakeencode1}\label{fig:snakedecode1}
% \end{figure*}

% Since the identifier hiding function in \LMKPM{} (see \cref{fig:MKPM}) is injective and deterministic for repeating identifiers, 
% this linkage is preserved in the protocol leakage $\leak_{\setT'}$, where $(\leak_{\setT'}, \leakV) = \leak(\setT', \setV)$.
% The adversary can therefore find the start of the linked list by finding the (unique) record whose $j^*$-th identifier
% is not appended to any other record and traverse the linked list to recover the original ordering of $\setT$.
% See procedure \snakerecoverUnique{} in \cref{fig:snakedecode1}.
% The procedure takes the target set leakage $\leak_{\setT'}$ and $j^*$ and 
% outputs the elements of $\leak_{\setT'}$ as a list $(t^*_1, \dots, t^*_n)$ such that $t^*_i$
% has index $i$ in the linked list.
% Having matched the hidden records in $\leak_{\setT'}$ with the plain records in $\setT'$,
% we can then apply the usual substitution-based transformation to infer a maximum reconstruction from \leakV{}.
% The full attack is displayed in \cref{fig:snakeattack}.

% \begin{figure}
% 	\centering
% 	\begin{pcvstack}[boxed, center]
% 		\procedure[linenumbering]{$\snakeattack^{\LMKPM(\cdot, \setV)}(\setT)$}{
% 			n \gets |\setT| \\
% 			\text{Fix some ordering $\setT = (t_1, \dots, t_n)$.}\\
% 			\pcif \text{$\lnot\exists j^*$ s.t. $|\{t_i[j^*] \setdsc i\in[n]\}| = n$} \pcthen \label{lin:distinctIDs_j*} \\
% 			\t 	\text{Pick $n$ distinct identifiers $\id_1, \dots \id_n \in \Ispace{}$.} \\
% 			\t 	\text{Prepend $\id_i$ to $t_i$ for all $i \in [n]$.}\\
% 			\t	j^* :=  1\\
% 			(t'_1, \dots, t'_n) \gets \snakeencodeUnique(j^*, (t_1, \dots, t_n)) \label{lin:snake_encode}\\
% 			(\UID, \MC, (\leak_{\setT'}, \leakV)) \sample \LMKPM(\{t'_1, \dots, t'_n\}, \setV) \label{lin:snake_evaluation}\\
% 			(t^*_1, \dots, t^*_n) \gets \snakerecoverUnique(\leak_{\setT'}, j^*) \label{lin:snake_decode}\pcskipln \\
% 			\pclinecomment{Construct substitution map $D$. }\\
% 			\text{Initialize empty dictionary $D$.} \\
% 			\pcfor i = 1, \dots, n \pcdo \label{lin:snake_D_start} \\
% 			\t	\pcfor j = 1, \dots |t| \pcdo \\
% 			\t	\t	D[t^*_i[j]] \gets t'_i[j] \label{lin:snake_D_end} \\
% 			\recVT \gets \dictInfer(\leakV{}, D) \\
% 			\pcreturn \recVT
% 		}
% 		\end{pcvstack}
% 		\caption{Snake attack with oracle access to the extended MKPM functionality $\LMKPM(\cdot, \setV)$.
% 		\setV{} is part of the oracle and unknown to the adversary.} \label{fig:snakeattack}
% \end{figure}

% \begin{theorem}
% 	For any sets of records \setT{} and \setV{}, 
% 	$\snakeattack^{\LMKPM(\cdot, \setV)}(\setT)$ produces a maximum reconstruction of \setV{} w.r.t. \setT.
% \end{theorem}
% \begin{proof}
% 	To prove the theorem, we prove the following proposition.
% 	The theorem follows directly from Proposition~\ref{thm:snake_D} and \cref{lem:dictinfer_MR}.
% \begin{proposition}
% 	\label{thm:snake_D}
% 	Let \setT{} and \setV{} be two sets of records and let $D$ be the substitution dictionary
% 	constructed during an invocation of $\snakeattack^{\LMKPM(\cdot, \setV)}(\setT)$.
% 	Moreover, let $f_C$ be the identifier hiding function sampled during the invocation of \leak{}
% 	in line~\ref{lin:snake_evaluation}.
% 	Then, we have $D[f_C(\id)] = \id$ for all $\id \in \Ispace{\setT}$.
% \end{proposition}
% \begin{proof} \textbf{of Proposition~\ref{thm:snake_D}}
% 	Let $\setT' := \{t'_1, \dots, t'_n\}$ where $t'_i$ is produced by \snakeencodeUnique{} in line~\ref{lin:snake_encode}
% 	and let $k' := |t'_i|$ for any $i\in [n]$\footnote{Recall that we assume that all records have the same length $\lambda$.
% 	Thus, $k' \in \{\lambda + 1, \lambda+2\}$}.
% 	We show that $D[f_C(\id)] = \id$ for all $\id \in \Ispace{\setT'}$.
% 	Since \snakeattack{} potentially adds fresh identifiers to records of \setT{}, but does not remove any, 
% 	we have $\Ispace{\setT}\subseteq\Ispace{\setT'}$, which implies the claim.
	
% 	Similar to the proof of \cref{thm:recenum_D}, we show that
% 	$t^*_i[j] = f_C(t'_i[j])$ for all $i\in[n]$ and all $j \in [k']$.
% 	The result then follows from the construction of $D$ in lines~\ref{lin:snake_D_start}~-~\ref{lin:snake_D_end}. 
% 	In particular, we will show that the linked list we create in \IdxToSeq{} is preserved in the leakage, 
% 	and that this linked list allows us to uniquely identify each record. 
% 	We show this by induction on $i$. 
% 	Let $j^*$ denote the index of a column whose identifiers are all unique (note that if such a column does not exist, it is created on line~\ref{lin:distinctIDs_j*}).

% 	For $i = 1$, observe that identifier $t_1[j^*]$ is not appended to any record in \snakeencodeUnique{}; 
% 	in other words, it is the only item in the linked list that doesn't have anything pointing to it.
% 	In contrast, for all $i' > 1$, identifier $t_{i'}[j^*]$ is appended to $t_{i'+1}$.
% 	$t'_1$ is therefore the only record in $\setT'$ whose $j^*$-th identifier is not contained in the last column, 
% 	i.e., $t'_1[j^*] \not \in \{t'_{i'}[k'] \setdsc i'\in[n]\}$.
% 	By the definition of $\LMKPM$, this property is preserved in the leakage $\leak_{\setT'}$.
% 	That is, the corresponding record in the leakage $h = (f_C(t'_1[1]), \dots, f_C(t'_1[k']))$
% 	is the only record in $\leak_{\setT'}$ such that such that $h[j^*]\not\in\{h_{i'}[k'] \setdsc i' \in [n]\}$.
% 	In line~\ref{lin:snake_recover_c} in \snakerecoverUnique{} we pick $h_c := h$.
% 	Thus, for all $j \in [k']$ we have that $h_c[j] = f_C(t'_1[j])$
% 	and, therefore  $t^*_1[j] = h_c[j] = f_C(t'_1[j])$, where the first equality follows from line~\ref{lin:snake_recover_assign_t_star}.

% 	Now assume that for some $i < n$ we have $t^*_i[j] = f_C(t'_i[j])$ for all $j \in [k']$. 
% 	We show that there is a unique record which $t'_i[k']$ is pointing to (that has not been traversed by the previous items in the linked list), 
% 	and that, once again this property is preserved in the leakage allowing for unique identification of the record.
% 	By construction in \snakeencodeUnique{}, we have $t'_{i+1}[j^*] = t'_i[k']$.
% 	Since all identifiers in the $j^*$-th column are distinct, 
% 	$t'_{i+1}$ is the only record in $\setT'$ that satisfies this condition.
% 	As before, since $f_C$ is injective, the corresponding record in the leakage 
% 	$h = (f_C(t'_{i+1}[1]), \dots, f_C(t'_{i+1}[k']))$
% 	is the only record that satisfies $h[j^*] = t^*_i[k']$.
% 	Moreover, all hidden identifiers in the $j^*$-th column of $\leak_{\setT'}$	are distinct. % $t'_{i+1}$ is the only record in $\setT'$
% 	Thus, the dictionary \loc{} in \SeqToIdx{} contains exactly the $n$ keys $\{f_C(t'_{i'}[j^*]) \setdsc i' \in [n]\} = \{f_C(t^*_{i'}[j^*]) \setdsc i' \in [n]\}$,
% 	which are mapped as $\loc[f_C(t'_{i'}[j^*])] = i'$.
% 	In line~\ref{lin:snake_recover_assign_t_star}, we pick $t^*_{i+1} := h_c$ for $c = \loc[t^*_i[k']]$, 
% 	implying $t^*_{i+1}[j^*] = t^*_i[k']$ and, thus, $t^*_{i+1} = h$.
% 	We therefore have $t^*_{i+1}[j] = h[j] = f_C(t'_{i+1}[j])$ for all $j \in [k']$.
% \end{proof}

% 	The statement follows from \cref{thm:snake_D} and \cref{lem:dictinfer_MR}.
% \end{proof}

% \textbf{Time and Query Complexity.} 
% Recall our assumptions that reading from, writing to, and checking for the presence of a key in a dictionary 
% requires an average-case constant time complexity.
% Then, \snakeencodeUnique{} has an asymptotic average-case time complexity of \bigO{n} for $n = |\setT|$,
% assuming \id{} in line \ref{lin:snake_encode_last_id} can be found in constant time\footnote{This is feasible with some additional assumptions on \setT. For instance, if \setT{} only contains email addresses, one may simply pick a single character for $\id$.}.
% The same holds for \snakerecoverUnique{}.
% In particular, $c$ in line~\ref{lin:snake_recover_c} can be found by inserting the last identifier of every record into a dictionary 
% and then checking for the presence of the $j^*$-th identifier of every record.
% Moreover, note that the loop in lines~\ref{lin:snake_recover_dowhile}~-~\ref{lin:snake_recover_dowhile_end}
% traverses the linked list once, executing exactly $n$ iterations.

% Let $\lambda$ denote the length of the largest record in \setT{} or \setV.
% Determining whether there exists a column in \setT{} that contains only unique identifiers
% can be done in time \bigO{\lambda n}, also assuming that dictionary insertions and lookups can be done in average-case constant time.
% Computing the dictionary $D$ requires at most $\lambda n$ write operations.
% Adding the complexity of \dictInfer{}, \bigO{\lambda m} with $m = |\setV|$,
% results in an overall average-case time complexity of \bigO{\lambda (n + m)} for \snakeattack.

% Like \toyattack{} and \recenumattack{}, \snakeattack{} requires only one protocol invocation.

% \subsection{Discussion and Mitigation}
% \label{sec:snake_mitigations}
% While requiring the same number of protocol invocations 
% and incurring a comparable computational complexity as the other attacks presented in this chapter, 
% \snakeattack{} stands out for requiring only minimal modifications to the target set to successfully carry out the attack.
% Since the records are extended by at most two identifiers, in contrast to $\log n$ (as required by \recenumattack{}),
% \snakeattack{} is also applicable to scenarios where the record length is constrained to a small constant.
% Meta's Private Computation Solution \cite{FBPCSRepo} serves as a concrete example, 
% where records are limited to at most four identifiers.
% If the target set already meets the maximum allowed record length, 
% the adversary can always rearrange the target set into shorter records and use the newly available space for the snake encoding.

% Since records are only extended by a small constant, limiting record lengths to mitigate against attacks, 
% as proposed previously, is ineffective.
% However, the chain-like structure of the modified target set, which enables recovering the original order of the identifiers,
% allows the victim to detect and mitigate the attack.
% To capture a broader variety of such linking strategies than just the approach we presented above, 
% we consider a graph $\graph(\setX) = (\vertSet_\setX, \edgeSet_\setX)$ for some set of records \setX{}, which contains one vertex for every record of the target set 
% and an edge between every pair of vertices whose corresponding records contain at least one common identifier.
% \begin{equation*}
% 	\vertSet_\setX := \{v_x \setdsc x \in \setX \}
% \end{equation*}
% \begin{equation*}
% 	\edgeSet_\setX := \{(v_x, v_{x'}) \setdsc \exists i, j. x[i] = x'[j] \}
% \end{equation*}
% As depicted in \cref{fig:snake_exmp}, the graph $\graph(\setT')$, which is generated from the modified target set $\setT'$ produced by \snakeencodeUnique{},
% only has a single connected component. 
% Furthermore, since the victims protocol leakage $\leak^P_{\setT'}$ (see \cref{fig:MKPM}) simply contains a shuffled and renamed copy of $\setT'$,
% $\graph(\setT')$ and $\graph(\leak^P_{\setT'})$ are isomorphic and thus have the same number of connected components. 
% The victim can therefore detect such linkage-based attack strategies and abort the protocol execution.
% This also defends against the record enumeration attack presented in the last section.
% However, one needs to be careful not to accidentally reject valid inputs, 
% as large connected components may occur naturally.
% The example of IP addresses behind a NAT device we mentioned in the last section also applies here.
