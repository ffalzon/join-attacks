\section{Applying the PSU-CA Attack to PrivateID}\label{sec:psuca-privateID-attack}

Meta's PrivateID protocol~\cite{PMC} implements the \emph{Single-Key Private Matching (SKPM)} functionality which we denote as $\calF_{\SKPM}$.

The functionality $\calF_{\SKPM}$ involves two parties, $P_1$ and $P_2$, each with input sets $X \subseteq \Ispace{}$ and $Y \subseteq \Ispace{}$, respectively. The functionality computes $Z := X \cup Y$ and assigns new identifiers to the elements of $Z$ by evaluating them under an injective random function, producing the set $\UID{}$. It then computes two maps $\MX: \UID{} \to X \cup \{\bot\}$ and $\MY: \UID{} \to Y \cup \{\bot\}$, which locally map elements in $\UID{}$ to the original elements in $X$ and $Y$ (or $\bot$ if the element only occurs in the other party's set). In the end, party $P_1$ receives the set of UIDs, $\UID{}$, and dictionary $\MX$; party $P_2$ receives $\UID{}$ and $\MY$.

See Figure~\ref{fig:SKPM_definition} for the formal definition.


\begin{figure}[H]
	\fbox{%
    \begin{minipage}[t]{0.98\columnwidth}
		\textbf{Parameters:} Two parties, $C$ and $P$.\\
		\noindent\textbf{Functionality:} $\calF_{\SKPM}(X, Y)$: 
        \hfill \begin{enumerate}
			\item Receive the input from party $P_1$: $X = \{x_i \setdsc i \in [n]\}$
			\item Receive the input from party $P_2$: $Y = \{y_i \setdsc i \in [m]\}$
			\item Compute $Z := X \cup Y$.
			\item Sample $f \sample \InjFuncs[\Ispace, \calG]$.
			\item Compute dictionaries
            $\MX : \calG \to X \cup \{\bot\}$ and $\MY : \calG \to Y \cup \{\bot\}$ s.t. for all $z \in Z$:
			\[
				\MX[f(z)] = \begin{cases*}
					z & if $z \in X$ \\
					\bot & otw.
				\end{cases*} \\
            \]
            \[
				\MY[f(z)] = \begin{cases*}
					z & if  $z \in Y$ \\
					\bot & otw.
				\end{cases*}
            \]
		\item Compute $\UID \gets \{f(z) \setdsc z \in Z\}$.
		\item Output:
            \begin{itemize}
                \item Send $\UID, \MX$ to $P_1$.
                \item Send $\UID, \MY$ to $P_2$.
			\end{itemize}
	\end{enumerate}  
    \end{minipage}
    }
	\caption{The Single-Key Private Matching functionality $\calF_{\SKPM}$, implemented by Meta's PrivateID protocol \cite{PMC}.}\label{fig:SKPM_definition}
\end{figure}


Since each element $z \in Z = X \cup Y$ is mapped to its unique identifier under an injective function $f$, the relationship $|\UID| = |X \cup Y|$ holds for any sets $X$ and $Y$. Consequently, we can directly apply $\PSUCAattack{}$ to $\calF_{\SKPM}$ to recover $X \cap Y$ in at most $|X| + 1$ protocol invocations. Guo et al.~\cite{USENIX:GHLWJL22} examined PrivateID in the context of their intersection recovery attack on $\PSICA{}$, where they recover $X \cap Y$ in at most $|X|$ protocol invocations. However, their approach relies on the fact that PrivateID leaks $|X \cap Y|$ during protocol execution, whereas our attack depends solely on the functionality output.

Since the $\PSUCAattack{}$ attack applies directly, we omit the pseudocode.
