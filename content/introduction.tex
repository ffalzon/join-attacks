\begin{table*}[h!]
\centering
\renewcommand{\arraystretch}{1.1}
\begin{tabular}{c c c c c c c}
\toprule
\textbf{Functionality} & \textbf{SOTA Protocols} & \textbf{Attacks} &  \textbf{Recovery Goal}  & \textbf{\# Queries} & \textbf{Adaptive}  \\
\midrule
\multirow{2}{*}{PSI-Sum}  & \multirow{2}{*}{PJC~\cite{EuroSP:IKNPSS20,AC:LPRST21}} 
       & Jiang et al.~\cite{NDSS:JiaDuYan24} &  Intersection &  &  \\ \cline{3-6}
    &  & Falzon \& Tang~\cite{USENIX:FalTan25} &  Key-value  &  &  \\
\hline
\hline
\multirow{3}{*}{PSI-Cardinality}  & \multirow{3}{*}{\shortstack{PJC~\cite{EuroSP:IKNPSS20,AC:LPRST21}\\Catalic~\cite{AC:DuoPhaTri20}}} 
       & Guo et al.~\cite{USENIX:GHLWJL22} &   &  & \\ \cline{3-6}
    &  & Jiang et al.~\cite{NDSS:JiaDuYan24} &   &   &  \\ \cline{3-6}
    &  & Falzon \& Tang~\cite{USENIX:FalTan25} &  $T\cap Y$  &  &   \\
\hline
\hline
PSI-Inner-Product & PJC~\cite{AC:LPRST21}  & Falzon \& Tang~\cite{USENIX:FalTan25} & Key-value  & 2 &   \\
\hline
Join (Union) & PMC~\cite{PMC},  MK-PMC~\cite{MKPMC}  & \textbf{This work} & Intersection  & 2 &   \\
%Join (Union) \& Secret Share Associated Data & PS$^3$I & 2 & $T\cap Y$  &  &   \\
%\hline
% Join (Left-Join) \& Secret Share Associated Data & DPMC &  \textbf{This work}  &  Input-mal. &  & \\
% Join (Left-Join) \& Secret Share Associated Data & D$_s$PMC & \textbf{This work}  & Input-mal.  &   & \\
\bottomrule
\end{tabular}
\vspace{3mm}
\caption{An overview of PSI functionalities and their corresponding attacks. Each attack operates in the input-malicious model, where the adversary behaves honestly according to the protocol specification but may provide maliciously chosen inputs. Let $T$ be the target set of values that the adversary wishes to learn, and $Y$ be the other party's input.}
\label{tab:protocol_comparison}
\end{table*}




\section{Introduction}
\label{sec:introduction}
% \tianxin{Main message: Understand the connections between these primitives,
% and the security implications using cryptanalysis.}

%% \ff{incorporate this somewhere in the introduction, potentially}

Recent focus of cryptographic research has shifted from conventional
all-or-nothing security to allowing secure computation on private data. This
change is reflected by a plethora of work on secure two-party computation (2PC)
and multi-party computation (MPC). In particular, multiple functionalities have
been introduced and deployed in the real world \cite{}. These include basic set
operations, such as private set intersection (PSI) and private set union (PSU).

PSI has been studied extensively in the literature, both on the constructive
side and analysis (more so in the related section) \cite{}. Our work instead
focuses to private set union (PSU), where two or more parties, each holding a
set to jointly compute the union of their sets without revealing anything beyond
the result~\cite{USENIX:DZBC25,USENIX:TBZCC25,USENIX:JSZG24,
  USENIX:ZCLZL23,USENIX:JSZDG22,CCS:GNBT25,CCS:ZCLPHW24,
  CCS:TCLZ23,ASIACCS:CSSW25,ASIACCS:BlaAgu12,AC:LiuGao23,AC:KRTW19,PoPETS:GaoNguTri24,
  ACNS:Frikken07,C:KisSon05}.

PSU has been proposed for use in numerous use cases, including the private
aggregation of IP block lists~\cite{USENIX:JSZDG22,AC:KRTW19}, hospital
data~\cite{USENIX:JSZDG22}, and network traffic data~\cite{USENIX:BSMD10}. The
security goal of PSU is to enable data sharing without revealing which records
appear frequently in the set intersection, commonly referred to as \emph{heavy
  hitters} \cite{}.

\heading{Private Multi-Key Join.} However, database operations are often more
complex than basic set operations, to enable richer forms of private data
analytics. For example, a company may be interested in grouping information
about the same individual across multiple databases. Each database record
contains a key, such as a name, together with multiple associated attributes,
such as age and credit score. If the key appears in the intersection of the
databases, one can then perform a union over the corresponding attributes to
complete the individual's personal information. This functionality is commonly
referred to as a \emph{database join}. Private database joins have been proposed
in prior work, such as \cite{AC:KRTW19} \cite{}, along with associated
cryptanalysis efforts \cite{}.

In real-world settings, more expressive join operations are often desired, such
as multi-key joins rather than single-key joins. We study the work \cite{MKPMC},
to understand the security guarantee provided by such a protocol. Since it is a
concrete protocol rather than a generic functionality, we also identify the
leakage that is present in the protocol itself. This analysis serves to clarify
what information can be learned beyond the intended output, and highlights
potential risks that should be carefully considered prior to deployment.

% However, there is a close relationship between the cardinalities of the union
% and the intersection of two sets, thus raising the question whether existing
% intersection recovery attacks against PSI-CA
% (e.g.,~\cite{USENIX:GHLWJL22,NDSS:JiaDuYan24,USENIX:FalTan25}) can be modified
% to work against PSU-CA. \ff{maybe move this paragraph up to intro}.

\heading{Threat Models and Recovery Goals.} Because there is an inherent
difference between analyzing generic PSU functionality and a concrete multi-key
join protocol, we divide our analysis into two categories: one that considers
only the generic functionality, and another that considers all information
available in the protocol execution. In the first category, although both PSU
and PSI aim to reveal nothing beyond the output, namely, the union or the
intersection, the security goals are subtle. In particular, PSU is intended to
hide which elements that lie in the intersection, whereas PSI hides the ones
that are not in the intersection. For PSU, an adversary may query an empty set,
which would completely reveal the other party's input set. Similarly, for PSI,
if one inputs the universe, they would obtain the other party's input set. The
usual justification is that thresholds or authorization mechanisms can be
imposed on the sets to restrict such queries. In this work, we aim to
understand, whether it is still possible to circumvent these mitigation
mechanisms and recover the underlying set. \tianxin{Maybe more specific, and add
  database abstraction.}

As part of our analysis on the multi-key join work~\cite{MKPMC}, we identify
several subtleties that are not fully addressed by their security proof sketch,
including repeated identifiers in the inputs and certain syntactic
inconsistencies. Moreover, the protocol reveals additional information beyond
simply the matched identifiers: party $C$ learns a renamed and shuffled copy of
the two input sets, enabling it to infer the matching pattern between them. To
address this, we separate the intended functionality (i.e., what we wish to
compute) from the additional leakage (i.e., information that may also be
inferred, such as set sizes).

\heading{Experimental Evaluation.} \tianxin{Add.}

\subsection{Contributions}
We present our contributions as follows.
\begin{itemize}[label=-]
\item We identify that PSU, when used as a building block for richer database
  functionalities such as database join, no longer guarantees hiding of the
  intersection. This limits how PSU can be safely used in larger and more
  complex contexts.
\item We demonstrate concrete attacks on a multi-key join protocol
  \cite{MKPMC}. We analyze both the intended functionality and the leakage
  present in the protocol. \tianxin{Add recovery results.} This highlights the
  importance of providing a detailed security proof, which is missing in the
  original work. In addition, we connect our analysis to follow-up work \cite{}
  and show that the non-collusion assumption in delegable settings requires
  special care; otherwise, the protocol would be subject to the same risks.
\item We present the relationship of security guarantees of PSU, PSI, and
  multi-key join by reductions between their corresponding attacks. \tianxin{Do
    we need this?}
\item Finally, we highlight the broader implications of our attacks and raise
  the need to reconsider common mitigation strategies, such as imposing
  thresholds or authorized inputs, in light of our findings that such mechanisms
  may be insufficient when adversaries can issue carefully crafted yet seemingly
  benign inputs.
\end{itemize}

\subsection{Related Work}

\tianxin{Move this to the front?}

\subsubsection{Protocols.} 
The seminal works of Freedman, Nissim, and Pinkas~\cite{EC:FreNisPin04} and
Agrawal, Evfimievski, and Srikant~\cite{SIGMOD:AgrEvfSri03} laid the foundations
for modern private set intersection and private database operations. Kissner and
Song~\cite{C:KisSon05} introduced a framework for privacy-preserving set
operations, including union, intersection, and element reduction. Since then, a
long line of work has proposed protocols for both PSI
(e.g.,\cite{C:PRTY19,USENIX:PinSchZoh14, CCS:CheLaiRin17, CCS:DonCheWen13,
  FC:DeCTsu10, ASIACCS:Kerschbaum12, CCS:RosTri21, C:ChaMia20}) and PSU
(e.g.,~\cite{USENIX:DZBC25,USENIX:TBZCC25,USENIX:JSZG24,USENIX:ZCLZL23,USENIX:JSZDG22,CCS:GNBT25,CCS:ZCLPHW24,CCS:TCLZ23}).
More recent constructions extend PSI to support private computation over
payloads associated with intersecting values, such as the
sum~\cite{EUROSP:IKNPSSRSY20} and inner
product~\cite{ASIACCS:CHIKT23,AC:LPRST21}.

Protocols supporting computations on tables as inputs have also been
proposed. Buddhavarapu et al.~\cite{PMC} proposed two protocols: (i) Private-ID,
which computes a set of pseudorandom universal identifiers corresponding to the
records in the union of the parties’ inputs, and (ii) PS$^3$I, which outputs
secret shares of the matched records. Subsequent work extends Private-ID to
support multi-key datasets~\cite{MKPMC} and to enable delegation of the
computation to an untrusted third party~\cite{PoPETS:MMTSBC24}.

Mohassel et al.~\cite{CCS:MohRinRos20} proposed a protocol for SQL-like join and
select queries and whose inputs and outputs can be secret shared between the
parties. Similarly, IDCloak~\cite{IDCloak} privately computes an $n$-party join
and outputs secret shares of the result. Asharov et al.~\cite{CCS:AHKNPT23}
proposed the first protocols that support JOIN and GROUP-BY (i.e., aggregation)
operations and are secure against malicious adversaries. Most recently, Lehmann
et al.~\cite{PoPETS:LehMouSid26} described a protocol that enables multiple
parties to provide a receiver with the inner joins over their respective
datasets. \ff{include explanation of why these works are relevant; do they consider a different threat model? Leave open for investigation in future works?}

\subsubsection{Attacks.} The study of attacks against set-operation
functionalities was initiated by Guo et al.~\cite{USENIX:GHLWJL22}, who
presented two intersection-recovery attacks against the Private Set Intersection
Cardinality (PSI-CA) and Private Set Intersection Sum (PSI-SUM)
functionalities. This work was extended by Jiang et al.~\cite{NDSS:JiaDuYan24},
who introduced new attacks on the same functionalities that required fewer
queries. Most recently, Falzon and Tang \cite{USENIX:FalTan25} proposed a series
of attacks on Google’s Private Join and Compute (PCJ)
functionality~\cite{AC:LPRST21}, which takes as input two key-value stores and
computes the inner product of the values associated with intersecting
keys. Their techniques draw on a range of approaches, including combinatorial
methods, maximum-likelihood estimation, and tools from signal processing.  In
another line of work, Zinkus et al.~\cite{USENIX:ZinCaoGre23} describe an
approach for automatically quantifying the leakage for a given functionality.