\begin{table*}[h!]
\centering
\renewcommand{\arraystretch}{1.1}
\begin{tabular}{c c c c c c c}
\toprule
\textbf{Functionality} & \textbf{SOTA Protocols} & \textbf{Attacks} &  \textbf{Recovery Goal}  & \textbf{\# Queries} & \textbf{Adaptive}  \\
\midrule
\multirow{2}{*}{PSI-Sum}  & \multirow{2}{*}{PJC~\cite{EuroSP:IKNPSS20,AC:LPRST21}} 
       & Jiang et al.~\cite{NDSS:JiaDuYan24} &  Intersection &  &  \\ \cline{3-6}
    &  & Falzon \& Tang~\cite{USENIX:FalTan25} &  Key-value  &  &  \\
\hline
\hline
\multirow{3}{*}{PSI-Cardinality}  & \multirow{3}{*}{\shortstack{PJC~\cite{EuroSP:IKNPSS20,AC:LPRST21}\\Catalic~\cite{AC:DuoPhaTri20}}} 
       & Guo et al.~\cite{USENIX:GHLWJL22} &   &  & \\ \cline{3-6}
    &  & Jiang et al.~\cite{NDSS:JiaDuYan24} &   &   &  \\ \cline{3-6}
    &  & Falzon \& Tang~\cite{USENIX:FalTan25} &  $T\cap Y$  &  &   \\
\hline
\hline
PSI-Inner-Product & PJC~\cite{AC:LPRST21}  & Falzon \& Tang~\cite{USENIX:FalTan25} & Key-value  & 2 &   \\
\hline
Join (Union) & PMC~\cite{PMC},  MK-PMC~\cite{MKPMC}  & \textbf{This work} & Intersection  & 2 &   \\
%Join (Union) \& Secret Share Associated Data & PS$^3$I & 2 & $T\cap Y$  &  &   \\
%\hline
% Join (Left-Join) \& Secret Share Associated Data & DPMC &  \textbf{This work}  &  Input-mal. &  & \\
% Join (Left-Join) \& Secret Share Associated Data & D$_s$PMC & \textbf{This work}  & Input-mal.  &   & \\
\bottomrule
\end{tabular}
\vspace{3mm}
\caption{An overview of PSI functionalities and their corresponding attacks. Each attack operates in the input-malicious model, where the adversary behaves honestly according to the protocol specification but may provide maliciously chosen inputs. Let $T$ be the target set of values that the adversary wishes to learn, and $Y$ be the other party's input.}
\label{tab:protocol_comparison}
\end{table*}






\section{Introduction}

There is a long line of work on Private Set Union (PSU), where two or more parties---each holding a set---jointly compute the union of their sets without revealing anything beyond the result~\cite{USENIX:DZBC25,USENIX:TBZCC25,USENIX:JSZG24, USENIX:ZCLZL23,USENIX:JSZDG22,CCS:GNBT25,CCS:ZCLPHW24, CCS:TCLZ23,ASIACCS:CSSW25,ASIACCS:BlaAgu12,AC:LiuGao23,AC:KRTW19,PoPETS:GaoNguTri24,
ACNS:Frikken07,C:KisSon05}. PSU has been proposed for use in numerous use-cases, including the private aggregation of IP block lists~\cite{USENIX:JSZDG22,AC:KRTW19}, hospital data~\cite{USENIX:JSZDG22}, and network traffic data~\cite{USENIX:BSMD10}, as well as supporting full private joins over databases~\cite{AC:KRTW19}. 

% However, there is a close relationship between the cardinalities of the union and the intersection of two sets, thus raising the question whether existing intersection recovery attacks against PSI-CA (e.g.,~\cite{USENIX:GHLWJL22,NDSS:JiaDuYan24,USENIX:FalTan25}) can be modified to work against PSU-CA. \ff{maybe move this paragraph up to intro}.



\ff{incorporate this someowhere in the introduction, potentially}
The original work~\cite{MKPMC} provides a proof sketch arguing security against a semi-honest adversary. However, several subtleties are not fully addressed, including repeated identifiers in the inputs and certain syntactic inconsistencies. Moreover, the protocol reveals additional information beyond simply the matched identifiers: party $C$ learns a renamed and shuffled copy of the two input sets, enabling it to infer the matching pattern between them. To address this, we separate the intended functionality (i.e., what we wish to compute) from the additional leakage (i.e., information that may also be inferred, such as set sizes).

\subsection{Contributions}


\subsection{Related Work} 

\subsubsection{Protocols.} 
The seminal works of Freedman, Nissim, and Pinkas~\cite{EC:FreNisPin04} and
Agrawal, Evfimievski, and Srikant~\cite{SIGMOD:AgrEvfSri03} laid the foundations for modern private set intersection and private database operations.
Kissner and Song~\cite{C:KisSon05} introduced a framework for privacy-preserving set operations, including union, intersection, and element reduction. Since then, a long line of work has proposed protocols for both PSI (e.g.,\cite{C:PRTY19,USENIX:PinSchZoh14, CCS:CheLaiRin17, CCS:DonCheWen13, FC:DeCTsu10, ASIACCS:Kerschbaum12, CCS:RosTri21, C:ChaMia20}) and PSU (e.g.,~\cite{USENIX:DZBC25,USENIX:TBZCC25,USENIX:JSZG24,USENIX:ZCLZL23,USENIX:JSZDG22,CCS:GNBT25,CCS:ZCLPHW24,CCS:TCLZ23}).
More recent constructions extend PSI to support private computation over payloads associated with intersecting values, such as the sum~\cite{EUROSP:IKNPSSRSY20} and inner product~\cite{ASIACCS:CHIKT23,AC:LPRST21}.

Protocols supporting computations on tables as inputs have also been proposed. Buddhavarapu et al.~\cite{PMC} proposed two protocols: (i) Private-ID, which computes a set of pseudorandom universal identifiers corresponding to the records in the union of the parties’ inputs, and (ii) PS$^3$I, which outputs secret shares of the matched records. Subsequent work extends Private-ID to support multi-key datasets~\cite{MKPMC} and to enable delegation of the computation to an untrusted third party~\cite{PoPETS:MMTSBC24}.

Mohassel et al.~\cite{CCS:MohRinRos20} proposed a protocol for SQL-like join and select queries and whose inputs and outputs can be secret shared between the parties. Similarly, IDCloak~\cite{IDCloak} privately computes an $n$-party join and outputs secret shares of the result. Asharov et al.~\cite{CCS:AHKNPT23} proposed the first protocols that support JOIN and GROUP-BY (i.e., aggregation) operations and are secure against malicious adversaries. Most recently, Lehmann et al.~\cite{PoPETS:LehMouSid26} described a protocol that enables multiple parties to provide a receiver with the inner joins over their respective datasets. 

\subsubsection{Attacks.} The study of attacks against set-operation functionalities was initiated by Guo et al.~\cite{USENIX:GHLWJL22}, who presented two intersection-recovery attacks against the Private Set Intersection Cardinality (PSI-CA) and Private Set Intersection Sum (PSI-SUM) functionalities. This work was extended by Jiang et al.~\cite{NDSS:JiaDuYan24}, who introduced new attacks on the same functionalities that required fewer queries. Most recently, Falzon and Tang \cite{USENIX:FalTan25} proposed a series of attacks on Google’s Private Join and Compute (PCJ) functionality~\cite{AC:LPRST21}, which takes as input two key-value stores and computes the inner product of the values associated with intersecting keys. Their techniques draw on a range of approaches, including combinatorial methods, maximum-likelihood estimation, and tools from signal processing.
In another line of work, Zinkus et al.~\cite{USENIX:ZinCaoGre23} describe an approach for automatically quantifying the leakage for a given functionality. 

