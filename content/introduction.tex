\begin{table*}[h!]
\centering
\renewcommand{\arraystretch}{1.1}
\begin{tabular}{r c c c c c c c}
\toprule
\textbf{Attacks} & \textbf{Functionality} &  \textbf{Recovery Goal}  & \textbf{\# Queries} & \textbf{Approx.} & \textbf{Adaptive} & \textbf{Auxiliary Data} \\
\midrule
Guo et al.~\cite{USENIX:GHLWJL22} & PSI-CA & Intersection &  &  & \checkmark &  \\ 
\hline
Jiang et al.~\cite{NDSS:JiaDuYan24} & PSI-CA & Intersection &  &  &  \checkmark & \\ 
Jiang et al.~\cite{NDSS:JiaDuYan24} & PSI-CA & Intersection &  & \checkmark &  \checkmark & \\ 
Jiang et al.~\cite{NDSS:JiaDuYan24} & PSI-Sum~\cite{EuroSP:IKNPSS20} & Intersection &   &  &  \checkmark& \\
\hline
Falzon \& Tang~\cite{USENIX:FalTan25} & \multirow{4}{*}{PJC~\cite{AC:LPRST21}}  & Key-value  & $O(k \log n)$ &  &  \checkmark &  \\
Falzon \& Tang~\cite{USENIX:FalTan25} &    & Key-value  & -- & \checkmark &   & Value Distribution \\
Falzon \& Tang~\cite{USENIX:FalTan25} &   & Key-value  & $2k$ &  &   & Intersection size $k$ \\
Falzon \& Tang~\cite{USENIX:FalTan25} &    & Key-value  & $\Theta(s \log(n/s))$ & \checkmark  &   &  \\
\hline
\textbf{This work} (Sec.~\ref{sec:psu-attack}) & PSU & Intersection  & 2 &  &   &  \\
\textbf{This work} (Sec.~\ref{sec:psu-ca-attack}) & PSU-CA & Intersection  & add bound &  &  \checkmark &  \\
\textbf{This work} & \multirow{4}{*}{MK-PMC~\cite{MKPMC}} &  Max-Set Recon. & 1 &  &  \checkmark &  \\
\textbf{This work} &  & Max-Set Recon.  & 1 &  &  \checkmark &  \\
\textbf{This work} &  & Max-Set Recon.  & 1 &  &  \checkmark &  \\
\textbf{This work} &  & Max-Set Recon.  & 1 &  &  \checkmark &  \\
\bottomrule
\end{tabular}
\vspace{3mm}
\caption{An overview of PSI functionalities and their corresponding attacks. Each attack operates in the input-malicious model, where the adversary behaves honestly according to the protocol specification but may provide maliciously chosen inputs. Let $T$ be the target set of values that the adversary wishes to learn, and $Y$ be the other party's input.}
\label{tab:protocol_comparison}
\end{table*}


\section{Introduction}

There is a long line of work on Private Set Union (PSU), where two or more parties---each holding a set---jointly compute the union of their sets without revealing anything beyond the result~\cite{USENIX:DZBC25,USENIX:TBZCC25,USENIX:JSZG24, USENIX:ZCLZL23,USENIX:JSZDG22,CCS:GNBT25,CCS:ZCLPHW24, CCS:TCLZ23,ASIACCS:CSSW25,ASIACCS:BlaAgu12,AC:LiuGao23,AC:KRTW19,PoPETS:GaoNguTri24,
ACNS:Frikken07,C:KisSon05}. PSU has been proposed for use in numerous use-cases, including the private aggregation of IP block lists~\cite{USENIX:JSZDG22,AC:KRTW19}, hospital data~\cite{USENIX:JSZDG22}, and network traffic data~\cite{USENIX:BSMD10}, as well as supporting full private joins over databases~\cite{AC:KRTW19}. 

\subsection{Prior Work}

\heading{Join Protocols.} 

\heading{Attacks.} The study of attacks against set-operation functionalities was initiated by Guo et al.~\cite{USENIX:GHLWJL22}, who presented two intersection-recovery attacks against the Private Set Intersection Cardinality (PSI-CA) and Private Set Intersection Sum (PSI-SUM) functionalities. This work was extended by Jiang et al.~\cite{NDSS:JiaDuYan24}, who introduced additional attacks on the same functionalities, including improvements over Guo et al.’s attack. Most recently, Falzon and Tang \cite{USENIX:FalTan25} proposed a series of attacks on Google’s Private Join and Compute (PCJ) functionality~\cite{AC:LPRST21}, which takes as input two key-value stores and computes the inner product of the values associated with intersecting keys. Their techniques draw on a range of approaches, including combinatorial methods, maximum-likelihood estimation, and tools from signal processing.

\todo{also cite~\cite{USENIX:ZinCaoGre23}}