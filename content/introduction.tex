\section{Introduction}

%https://www.usenix.org/system/files/sec22-jia.pdf
%  For example, it can be used for cyber
% risk assessment and management. Specifically, “Individual
% blacklists today suffer from several drawbacks that limit their
% accuracy in malicious source identification. ... Aggregating
% blacklists
% PSU can be employed
% to compute the union of cancer patients of different hospitals
% while hiding the identities of the patients who had cancer
% treatment at multiple hospitals, which involves patient privacy.
% Also, it can be used for privacy-preserving aggregation of
% network traffic statistics [4], merger of two Internet providers
% without revealing the information of their existing networks
% [3], and private database supporting full join [19]. 

\subsection{Prior Work}

The study of attacks on set-operation functionalities was initiated by Guo et al.~\cite{USENIX:GHLWJL22}, who presented two intersection-recovery attacks against the Private Set Intersection Cardinality (PSI-CA) and Private Set Intersection Sum (PSI-SUM) functionalities. This line of work was extended in 2024 by Jiang et al.~\cite{NDSS:JiaDuYan24}, who introduced additional attacks on the same functionalities, including improvements over Guo et al.’s attack. Most recently, Falzon and Tang \cite{USENIX:FalTan25} proposed a series of attacks on Google’s Private Join and Compute (PCJ) functionality~\cite{AC:LPRST21}, which takes as input two key-value stores and computes the inner product of the values associated with intersecting keys. Their techniques draw on a range of approaches, including combinatorial methods, maximum-likelihood estimation, and tools from signal processing.

\todo{also cite~\cite{USENIX:ZinCaoGre23}}