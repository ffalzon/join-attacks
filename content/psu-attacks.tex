\section{Analysis of PSU and PSU-CA}
\label{sec:psu-psuca}
While PSU and PSU-CA have received increasing attention, the information leaked
by their functionalties remains largely
unexplored. Frikken~\cite{ACNS:Frikken07} noted that submitting an empty set in
PSU fully reveals the peer's input and suggested abort-based mitigations. Jia et
al.~\cite{USENIX:JSZG24} further analyzed PSU leakage patterns, showing that Guo
et al.'s~\cite{USENIX:GHLWJL22} PSI-CA attack extends to certain PSU variants,
even under early-abort conditions. They proposed an ``enhanced'' PSU
functionality that delays leakage until protocol termination to strengthen
rate-limiting mitigations.

We demonstrate that such countermeasures are insufficient by introducing an
intersection-recovery attack against PSU that requires only two protocol
executions and which does not rely on the adversarial input set being empty or
disjoint from the victim set. We then prove that PSU-CA leaks at least as much
information as PSI-CA by constructing a reduction showing that any PSI-CA attack
with $q$ queries can be transformed into a PSU-CA attack with $q{+}3$
queries. Finally, in the next section, we show that our PSU-CA attack extends to Meta's $\calF_{\MKPM}$ functionality.

\subsection{Analysis of PSU}
\label{sec:psu-attack}
Recall that the input-malicious party $\tP_1$ knows the target set $T$ and can invoke $\calF_{\PSU}(;Y)$ on any set $X$. The adversary's goal is to recover $T \cap Y$. Our attack is based on the following equation. For any two sets $X$ and $Y$ we have 
\begin{equation}
  \label{eqn:PSU_basis}
  X \cap Y = (X \cup Y) \setminus (Y \setminus X) \setminus (X \setminus Y).
\end{equation} 

Both $X \cup Y$ and $Y \setminus X$ can be easily determined. The
former is the output of evaluating $\calF_{PSU}(X;Y)$, and the latter can be
computed as
$(X \cup Y) \setminus X = Y \setminus X$.
The challenge lies in determining $X \setminus Y$. To address
this, we make two additional observations.

\begin{proposition}\label{prop:PSU_difference_distributes}
  Let $X$ and $Y$ be sets and $X_1 \cup X_2$ be a partition of $X$. It holds that
  $X \setminus Y = (X_1 \setminus Y) \cup (X_2 \setminus Y).$
\end{proposition}

\begin{proposition}
  \label{prop:PSU_comp_partial_difference}
  Let $X$ and $Y$ be sets and let $X_1\cup X_2$ be a partition of $X$.
  Furthermore, let $Z_1 := X_1 \cup Y$ and $Z_2 := X_2 \cup Y$.  Then,
  $X_1 \setminus Y = X_1 \setminus Z_2$ and
  $X_2 \setminus Y = X_2 \setminus Z_1$.
\end{proposition}

The proofs can be found in Appendix~\ref{proof:PSU_difference_distributes} and~\ref{proof:PSU_comp_partial_difference}, respectively.

We could combine Propositions~\ref{prop:PSU_difference_distributes} and
\ref{prop:PSU_comp_partial_difference} with to obtain $X \cap Y$ with three
evaluations of $\PSU$. However, since
$X \cup Y = (X_1 \cup Y) \cup (X_2 \cup Y)$, the adversary can compute the full
union from the two partial unions, thus, saving one evaluation.  

Thus, by querying $\calF_{\PSU}(;Y)$ on two disjoint sets $X_1\cup X_2=T$ that form a partition of the target set, $\tP_1$ can recover the intersection. The pseudocode for the
attack \PSUattack{} is given in Figure~\ref{fig:PSU_attack} (Appendix~\ref{ap:psu-attack-pseudocode}).

\begin{theorem}
  Let $T$ be the target set and $Y$ be the recovery set. The attack $\PSUattack^{\calF_{PSU}(\cdot; Y)}(T)$ (Figure~\ref{fig:PSU_attack}) achieves intersection recovery (Def.~\ref{def:intersection-recovery}).
\end{theorem}
\begin{proof}
  The theorem follows from
  Propositions~\ref{prop:PSU_difference_distributes} and
  \ref{prop:PSU_comp_partial_difference}.
\end{proof}

This is a very general attack, which only requires that $\calF_{\PSU}$ be evaluated
twice on disjoint sets while the
recovery set $Y$ remains static. These assumptions are well within the security model of proposed PSU protocols. 


\subsection{PSU-CA Attack}\label{sec:PSUCA_attack}
There is a close relationship between the
cardinalities of the union and the intersection of two sets.  In this next subsection, we give a generic technique that transforms any intersection recovery attack against PSI-CA (e.g.,~\cite{USENIX:GHLWJL22}) into one against PSU-CA. Specifically, if $T$ is the target set and $Y$ is the recovery set, then the adversarial party $\tP_1$ is able to recover $T\cap Y$.

Let $X$ and $Y$ be two sets. Our
attack is based on the following relation between the set union and intersection cardinalities:

\begin{equation}\label{eqn:union_intersect_cardinalities}
  |X \cap Y| = |X| + |Y| - |X \cup Y|
\end{equation}

In words, the attacker can compute the cardinality of the intersection if it knows
the cardinalities of the recovery set and the union.
This enables the adversary to translate between the PSU-CA and PSI-CA
functionalities, thus, allowing it to leverage any intersection recovery attack for PSI-CA against PSU-CA.

One issue remains: the adversary does not know $|Y|$ in the general case.
While some PSU-CA protocols may leak this information, not all do.   We therefore present
a general method to determine $|Y|$ with three $\PSUCA$ evaluations that avoids the trivial solution of querying the empty set, ie.., $\calF_{\PSUCA}(\emptyset; Y)$.

\begin{proposition}\label{prop:PSUCA_comp_m} Let $X$ and $Y$ be two sets and let
$X_1\cup X_2$ be a partition of $X$.  Moreover, let $z:= |X \cup Y|$, $z_1 := |X_1
\cup Y|$ and $z_2 := |X_2 \cup Y|$. Then, we have $|Y| = z_2 - (z - z_1)$.
\end{proposition}


We give the proof in Appendix~\ref{proof:PSUCA_comp_m}.
Thus, with three additional $\calF_{\PSUCA}$ evaluations at the start of the attack, the adversary can simulate the $\calF_{\PSICA}$ functionality. In particular, for any PSI-CA attack, the adversary can replace each 
query $z\gets \calF_{\PSICA}(X;Y)$ with $z'\gets \calF_{\PSUCA}(X;Y)$ and then compute $z$ as $|X|+|Y|-z'$.

We state our result in the following theorem:

\begin{theorem}\label{thm:PSICA-to-PSUCA}
  Let $T$ be the target set and $Y$ be the recovery set.
  For every adversary $\adv$ such that $\adv^{\PSICA(\cdot; Y)}(T) = T \cap Y$
  which makes $q\in\N$ $\PSICA$ queries,
  there is an an adversary $\bdv$ with oracle access to $\calF_{\PSUCA}(\cdot, Y)$
  which recovers $T \cap Y$ and makes $q + 3$ $\PSUCA$ queries.
\end{theorem}

The proof is given in Appendix~\ref{proof:PSICA-to-PSUCA}.
In Appendix~\ref{attack:PSUCA}, we concretely describe and analyze $\PSUCAattack$---an attack that uses the transformation described above to turn the PSI-CA attack from Guo et al.\cite{USENIX:GHLWJL22} (See Appendix~\ref{ap:guo_attack} for an overview) to one against PSU-CA.  




