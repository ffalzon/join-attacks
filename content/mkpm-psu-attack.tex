
\section{Analysis of the MKPM Functionality}\label{sec:PSUCA_to_matching}
In this section, we show how to apply the $\PSUCAattack{}$ attack to the functionality $\calF_{\MKPM}$---with some additional assumptions---to achieve MRR (Definition~\ref{def:matchable_records_recovery}).
In Appendix~\ref{sec:psuca-privateID-attack} we also explain how $\PSUCAattack{}$ can be applied to the single key private matching functionality $\calF_{\SKPM}$ realized by PrivateID~\cite{PMC}.


Recall that the adversary is given a target set of records $\setT\subseteq\Ispace{}^n$, and can evaluate $\calF_{\MKPM}(\setX, \setY)$
for arbitrarily many inputs $\setX$, while the set $\setY$ remains static.
Its goal is to determine the set $\setT \sqcap \setY$.

$\mathcal F_{\mathrm{MKPM}}$ outputs the set~$\UID$ of universal identifiers
assigned to all records held by the parties. Rather than assigning a common UID
to two identical records, it applies a matching logic (see
Figure~\ref{fig:match_logic}) and assigns a common UID to each pair of matched records,
while assigning a distinct UID to every unmatched record in~$\setT$ and~$\setY$. Thus,
using $|\UID|$ as the cardinality for carrying out the
$\PSUCAattack{}$ appears to be a natural approach for identifying matching records
in~$\setT$.

 
Recall that the $\PSUCAattack{}$ attack relies on the observation that $|X \cap Y| = |X_1 \cap Y| + |X_2 \cap Y|$, where $X$ and $Y$ are arbitrary sets and $X_1, X_2$ form a partition of $X$.
Applying Eq.~\ref{eqn:union_intersect_cardinalities}, this is equivalent to $|X \cup Y| = |X_1 \cup Y| + |X_2 \cup Y| - |Y|$.
Translating this to the multi-key setting, let $\setX \subseteq \setT$ and let $\setX_1, \setX_2$ be a partition of $\setX$.
Moreover, let $\UID$, $\UID_1$, and $\UID_2$ be the result from evaluating $\calF_{\MKPM}(\setX, \setY)$ and $\MKPM(\setX_i, \setY)$ for $i \in \{1,2\}$ respectively.
Then we require the following relation to hold:
\begin{equation}
	\label{eqn:MKPM_PSUCA_condition}
	|\UID| = |\UID_1| + |\UID_2| - |\setY|
\end{equation}

Unfortunately, this does not hold for general inputs for two reasons.
First, the MKPM functionality shuffles the inputs before evaluating the matching logic on them. As a result, the record matching logic does not consistently match the same number of records in different protocol runs for some fixed inputs, i.e., different evaluations of $\calF_\MKPM$ with the same inputs may result in $\UID$ sets with different cardinalities.
Second, the matching logic violates Eq.~\ref{eqn:MKPM_PSUCA_condition} even in cases where shuffling the inputs has no effect on the former.

Consider the example depicted in Figure~\ref{fig:MPMK_PSUCA_counterexample}.
% On the left, we show an example of a target set \setT{} and victim set \setV{}, 
% where two records in the target set share an identifier with the same record in the victim set.
% We call this a \emph{many-to-one relation}.
% Recall that the matching logic uses the order of identifiers in \setT{} as priorities to resolve such many-to-one relations.
% That is, it matches the second target record with the victim record, since it shares its first identifier $\id_3$ with the victim record,
% as opposed to the first target record, which shares its second identifier with the victim record. The latter remains unmatched.
% The resulting set \UID{} therefore contains two elements: 
% one uid for the two matched records and one for the unmatched first target record.
% On the right, we show a partition of \setT{}, matched against the same victim set \setV{} (in two separate protocol invocations).
% Now isolated and lacking competition from the other record, both target records are matched.
% The resulting sets $\UID_1$ and $\UID_2$ therefore both contain one uid.
% That is, we have $|\UID| = |\UID_1| + |\UID_2| \neq |\UID_1| + |\UID_2| - |\setV|$, which violates the requirement mentioned above.

\begin{figure}
	\centering
	{\def\linecorr{-.1pt} % Reduced correction for tighter placement
	\def\rowspace{1cm} % Adjust row space
	\def\colspace{3cm} % Widen the figures
	\def\setspace{0.2cm} % Reduce space between figures
	\def\setvertspace{.5cm} % Adjust vertical space between figure rows
	\def\minwidthid{0.8cm} % Slightly smaller blocks
	\def\blockminheight{1.8em} % Slightly smaller blocks
	\begin{tikzpicture}
		[block/.style={draw,minimum width=#1, minimum height=\blockminheight,font={\small}},
		block/.default=10em,high/.style={},auto,
		node distance=0, % Initially 0 to control spacing
		>=Stealth]
		
		% Top Figure
		% Nodes
		\node[block=\minwidthid,fill=trow-dark] (alice1) {$\id_1$};
		\node[block=\minwidthid,fill=trow-dark, right=\linecorr of alice1] (alice2) {$\id_2$};

		\node[block=\minwidthid,fill=trow-light, below=\rowspace of alice1] (bob1) {$\id_3$};
		\node[block=\minwidthid,fill=trow-light, right=\linecorr of bob1] (bob2) {$\id_4$};

		\coordinate (mid) at ($(alice2)!0.5!(bob2)$);

		\node[block=\minwidthid,fill=trow-dark, right=\colspace of mid] (vic1) {$\id_3$};
		\node[block=\minwidthid,fill=trow-dark, right=\linecorr of vic1] (vic2) {$\id_2$};


		% Connections
		\draw[dashed] (alice2) to (vic1);
		\draw (bob2) to (vic1);

		% Labels and descriptions
		\coordinate (center1) at ($(alice2.south east)!0.5!(vic1.west)$);
		
		\node[above=.3cm of alice1.east] (Tlabel) {$T$};
		\node[above=.3cm of vic1.east] (Vlabel1) {$Y$};
		\node[below=.8cm of center1] (UIDlabel) {$|\UID| = 2$};

		% Bottom Figure
		% Adjust vertical placement by node distance and below distance
		\coordinate (bottom_start) at ($(bob1.south) + (0,-\setspace)$);

		\node[block=\minwidthid,fill=trow-dark, below=\setvertspace of bottom_start] (alicesub1) {$\id_1$};
		\node[block=\minwidthid,fill=trow-dark, right=\linecorr of alicesub1] (alicesub2) {$\id_2$};
		\node[block=\minwidthid,fill=trow-dark, right=\colspace of alicesub2] (vicsub1) {$\id_3$};
		\node[block=\minwidthid,fill=trow-dark, right=\linecorr of vicsub1] (vicsub2) {$\id_2$};

		\node[block=\minwidthid,fill=trow-dark, below=\setvertspace of alicesub1] (bobsub1) {$\id_3$};
		\node[block=\minwidthid,fill=trow-dark, right=\linecorr of bobsub1] (bobsub2) {$\id_4$};
		\node[block=\minwidthid,fill=trow-dark, right=\colspace of bobsub2] (vicsub3) {$\id_3$};
		\node[block=\minwidthid,fill=trow-dark, right=\linecorr of vicsub3] (vicsub4) {$\id_2$};

		% Connections
		\draw (alicesub2) to (vicsub1);
		\draw (bobsub2) to (vicsub3);

		% Labels and descriptions
		\coordinate (center2) at ($(alicesub2.east)!0.5!(vicsub1.west)$);
		\coordinate (center3) at ($(bobsub2.east)!0.5!(vicsub3.west)$);

		\node [above=.3cm of alicesub1.east] (Tlabel2) {$T_1$};
		\node [above=.3cm of vicsub1.east] (Vlabel2) {$Y$};
		\node [below=0cm of center2] (UIDlabel1) {$|\UID_1| = 1$};

		\node [above=.3cm of bobsub1.east] (Tlabel3) {$T_2$};
		\node [above=.3cm of vicsub3.east] (Vlabel3) {$Y$};
		\node [below=0cm of center3] (UIDlabel2) {$|\UID_2| = 1$};
\end{tikzpicture}}
	\caption{Example of a target set $T$ and victim set $Y$, where $\PSUCAattack{}$ cannot be applied.
	Lines between records indicate potential matches based on matching identifiers. 
	Solid lines denote the matched records chosen by the matching logic.
	Clearly, $|\UID| \neq |\UID_1| + |\UID_2| - |Y|$.}
	\label{fig:MPMK_PSUCA_counterexample}
\end{figure}




% To avoid such cases, we formalize a restriction on the input sets that 
% -- if satisfied -- allows us to carry out the attack.
% Conveniently, the opposite case, where one record in the target set shares identifiers with multiple records in the victim set,
% does not lead to such problems.
% It therefore suffices to exclude many-to-one relations.

% \begin{definition}[Isolated Set]\label{def:isolated_set}
% 	Let \setX{} and \setY{} be two sets of records $\setX, \setY \subseteq \Ispace{}^{\leq \lambda}$ for some $\lambda\in\N$.
% 	We say \setX{} is \emph{$\setY$-isolated}, 
% 	if for all $v\in \setY$ we have $|\{t \in \setX \setdsc \exists i, j. \; t[i] = v[j]\}| \leq 1$.
% \end{definition}

% This additional constraint on the protocol inputs excludes some realistic scenarios.
% For instance, if the sets contain IP addresses
% -- IPv4 addresses behind a Network Address Translation (NAT) device in particular --
% such many-to-one relations are likely to occur, as users in the same local network share an address.
% However, this does not exclude the scenarios outlined in \cref{sec:threat_model},
% where the adversary chooses a specific range of identifiers as its target set, 
% such as all phone numbers of a certain area code or all email addresses belonging to some domain.
% Furthermore, scenarios where users possess multiple identifiers of some type,
% e.g., a private and a work email address, 
% but do not share them with other people, are still admitted as well, 
% provided that each user's identifiers only occur in at most one record in the target set.
% \cref{def:isolated_set} conveniently resolves both issues outlined above,
% which we prove in the following two lemmas.

% \begin{lemma}
% 	\label{lem:set_isolation_determinism}
% 	Let $\setX, \setY \subseteq \Ispace{}^{\leq \lambda}$ be two sets of records. 
% 	If \setX{} is \setY-isolated, 
% 	the cardinality of $\UID$ output by $\MKPM(\setX, \setY)$ (\cref{fig:MKPM}) is consistent across multiple evaluations.
% \end{lemma}
% \begin{proof}
% 	For any $x \in \setX$, let $\setY_x$ denote the records of \setY{} that share at least one identifier with $x$,
% 	i.e., $\setY_x := \{y \in \setY \setdsc \exists i, j \in \N : x[i] = y[j]\}$.
% 	If \setX{} is $\setY$-isolated, we have that $\setY_x \cap \setY_{x'} = \emptyset$ for all $x, x' \in \setX$ with $x \neq x'$. 
% 	Therefore, there exist no two records in \setX{} that could be matched with the same record in \setY{}.
	
% 	We first consider sets $\setY_x$ that are not empty.
% 	Since $x$ can only be matched with some $y \in \setY_x$ and all $y \in \setY_x$ can only be matched with $x$,
% 	\match{} (\cref{fig:match_logic}) will match $x$ with exactly one $y_x \in \setY_x$,
% 	which are assigned the same uid (line~\ref{lin:match_common_uid}).
% 	All other $y \in \setY_x$, i.e., $y \neq y_x$, will remain unmatched and are assigned their own uid (line~\ref{lin:match_unmatched_P_uid}).
% 	Furthermore, all $y \in \setY$ which do not belong to any $\setY_{x'}$ for any $x' \in \setX$ will also remain unmatched and are assigned their own uid (also line~\ref{lin:match_unmatched_P_uid}).
% 	Lastly, all $x\in\setX$ for which $\setY_x = \emptyset$ will remain unmatched and receive their own uid (line~\ref{lin:match_unmatched_C_uid}). 
	
% 	Thus, \UID{} contains one uid for each $y\in \setY$ and one uid for each unmatched $x\in \setX$, i.e.,
% 	$|\UID| = |\setY| + |\{x \in \setX \setdsc \setY_x = \emptyset\}|$.
% 	This is independent of the order of the records in \setX{} and \setY{} and
% 	thus also independent of the shuffling done before the matching step in \MKPM{}.
% 	Since \MKPM{} only uses randomness to shuffle the inputs, we have proven the lemma.
% \end{proof}

\begin{lemma}
	\label{lem:set_isolation_inference}
	Let $X$ and $V\subseteq \Ispace{}^{\leq \lambda}$ be two sets of records and let $X_1$ and $X_2$
	be a partition of $X$. 
	Moreover, let $(\UID, \MX) \sample \MKPM(X, Y)$ and $(\UID_i, M_{X,i}) \sample \MKPM(X_i, Y)$ for $i \in \{1,2\}$.
	If $X$ is $V$-isolated, we have $|\UID| = |\UID_1| + |\UID_2| - |V|$.
\end{lemma}

% \begin{proof}
% 	We show this by induction on $|\setX_1|$.
% 	If $|\setX_1| = 0$ all records of $\setY$ are unmatched and, thus, $|\UID_1| = |Y|$.
% 	Note that $\setX_2 = \setX$ and thus we have $|\UID| = |\UID_2| = |\UID_2| + |\UID_1| - |Y|$.
% 	In the first equality, we use \cref{lem:set_isolation_determinism}. 

% 	Assume $|\setX_1| = n_1$ for some $n_1 > 0$ and $|\UID| = |\UID_1| + |\UID_2| - |\setY|$.
% 	Since $\setX_1$ and $\setX_2$ form a partition of $\setX$, we must move a record from $\setX_2$ to $\setX_1$
% 	to achieve $|\setX_1| = n_1+1$. Note that $|\UID|$ remains unchanged due to \cref{lem:set_isolation_determinism}.
% 	Let $x \in \setX_2$ and let $\setX'_1 := \setX_1 \cup \{x\}$ and $\setX'_2 := \setX_2 \setminus \{x\}$.
% 	Moreover, let $\UID'_1$ and $\UID'_2$ be the sets of UIDs resulting from evaluating $\MKPM(\setX'_1, \setY)$ and $\MKPM(\setX'_2, \setY)$. 
% 	We distinguish two cases.
% 	If no identifier of $x$ occurs in $\setY$, $x$ was assigned its own UID.
% 	Therefore, $|\UID'_1| = |\UID_1| + 1$ and $|\UID'_2| = |\UID_2| - 1$, which implies the claim.
	
% 	For the second case, assume $x$ shares some identifiers with $n^*$ records of $\setY$.
% 	Call this set $\setY_x$.
% 	Since \setX{} is \setY-isolated, all records $y \in \setY_x$ only share identifiers with $x$,
% 	but no other $x'\in \setX_2$.
% 	Thus, $x$ is matched with some $y^* \in \setY_x$,
% 	i.e., $x$ and $y^*$ are assigned the same $\uid \in \UID_2$.
% 	After removing $x$ from $\setX_2$, $y^*$ will still be assigned some $\uid' \in \UID'_2$, 
% 	which it does not share with any $x' \in \setX'_2$, again since $\setX$ is \setY-isolated. 
% 	Therefore, $|\UID_2| = |\UID'_2|$. 
% 	Similarly, since $x\not\in \setX_1$, no $y\in \setY_x$ and $x' \in \setX_1$ are assigned the same $\uid \in \UID_1$.
% 	By the definition of $\setY_x$ and since \setX{} is \setY-isolated, $x$ will be assigned the same $\uid\in \UID'_1$
% 	as some $y\in\setY_x$ after being added to $\setX_1$, implying $|\UID_1| = |\UID'_1|$.
% 	The claim then follows trivially.
% \end{proof}

With the necessary properties of isolated input sets established, we can assert the correctness of our adaptation:

\begin{theorem}\label{thm:psu-mkpm:correctness}
	Let $T,V\subseteq \Ispace{}^n$ are sets of records.
	If $T$ is $Y$-isolated, then 
	$\PSUCAattack^{\calF_{\MKPM}(\cdot, Y)}(T)$ recovers $T \sqcap Y$.
\end{theorem}

The proof can be find in Appendix~\ref{sec:psu-mkpm:correctness}.

\heading{Time and query complexity.}
The above adaptations consist of translating the set union cardinality to the multi-key scenario.
In particular, the elements of $X$ that lie in the intersection $X\cap Y$ are analogous
to the matchable records in $T$.
The $\PSUCAattack{}$ applied to $\calF_{\MKPM}$ therefore has the same worst-case time complexity and maximum number of protocol invocations as when the attack is applied to $\calF_{\PSUCA}$.
